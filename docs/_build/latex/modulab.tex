%% Generated by Sphinx.
\def\sphinxdocclass{report}
\documentclass[letterpaper,10pt,english]{sphinxmanual}
\ifdefined\pdfpxdimen
   \let\sphinxpxdimen\pdfpxdimen\else\newdimen\sphinxpxdimen
\fi \sphinxpxdimen=.75bp\relax
\ifdefined\pdfimageresolution
    \pdfimageresolution= \numexpr \dimexpr1in\relax/\sphinxpxdimen\relax
\fi
%% let collapsible pdf bookmarks panel have high depth per default
\PassOptionsToPackage{bookmarksdepth=5}{hyperref}

\PassOptionsToPackage{booktabs}{sphinx}
\PassOptionsToPackage{colorrows}{sphinx}

\PassOptionsToPackage{warn}{textcomp}
\usepackage[utf8]{inputenc}
\ifdefined\DeclareUnicodeCharacter
% support both utf8 and utf8x syntaxes
  \ifdefined\DeclareUnicodeCharacterAsOptional
    \def\sphinxDUC#1{\DeclareUnicodeCharacter{"#1}}
  \else
    \let\sphinxDUC\DeclareUnicodeCharacter
  \fi
  \sphinxDUC{00A0}{\nobreakspace}
  \sphinxDUC{2500}{\sphinxunichar{2500}}
  \sphinxDUC{2502}{\sphinxunichar{2502}}
  \sphinxDUC{2514}{\sphinxunichar{2514}}
  \sphinxDUC{251C}{\sphinxunichar{251C}}
  \sphinxDUC{2572}{\textbackslash}
\fi
\usepackage{cmap}
\usepackage[T1]{fontenc}
\usepackage{amsmath,amssymb,amstext}
\usepackage{babel}



\usepackage{tgtermes}
\usepackage{tgheros}
\renewcommand{\ttdefault}{txtt}



\usepackage[Bjarne]{fncychap}
\usepackage{sphinx}

\fvset{fontsize=auto}
\usepackage{geometry}


% Include hyperref last.
\usepackage{hyperref}
% Fix anchor placement for figures with captions.
\usepackage{hypcap}% it must be loaded after hyperref.
% Set up styles of URL: it should be placed after hyperref.
\urlstyle{same}

\addto\captionsenglish{\renewcommand{\contentsname}{Contents:}}

\usepackage{sphinxmessages}
\setcounter{tocdepth}{1}



\title{MODULAB}
\date{Nov 18, 2025}
\release{0.1.1}
\author{Silas Hörz}
\newcommand{\sphinxlogo}{\vbox{}}
\renewcommand{\releasename}{Release}
\makeindex
\begin{document}

\ifdefined\shorthandoff
  \ifnum\catcode`\=\string=\active\shorthandoff{=}\fi
  \ifnum\catcode`\"=\active\shorthandoff{"}\fi
\fi

\pagestyle{empty}
\sphinxmaketitle
\pagestyle{plain}
\sphinxtableofcontents
\pagestyle{normal}
\phantomsection\label{\detokenize{index::doc}}


\sphinxstepscope


\chapter{Log Manager}
\label{\detokenize{modulab/log_manager:module-modules.log.LogManager}}\label{\detokenize{modulab/log_manager:id1}}\label{\detokenize{modulab/log_manager:log-manager}}\label{\detokenize{modulab/log_manager::doc}}\index{module@\spxentry{module}!modules.log.LogManager@\spxentry{modules.log.LogManager}}\index{modules.log.LogManager@\spxentry{modules.log.LogManager}!module@\spxentry{module}}\index{LogManager (class in modules.log.LogManager)@\spxentry{LogManager}\spxextra{class in modules.log.LogManager}}

\begin{fulllineitems}
\phantomsection\label{\detokenize{modulab/log_manager:modules.log.LogManager.LogManager}}
\pysigstartsignatures
\pysigline
{\sphinxbfcode{\sphinxupquote{\DUrole{k}{class}\DUrole{w}{ }}}\sphinxcode{\sphinxupquote{modules.log.LogManager.}}\sphinxbfcode{\sphinxupquote{LogManager}}}
\pysigstopsignatures
\sphinxAtStartPar
Bases: \sphinxcode{\sphinxupquote{QObject}}

\sphinxAtStartPar
Verwaltet das Logging für die gesamte Anwendung.

\sphinxAtStartPar
Diese Klasse ist als zentraler Dienst konzipiert. Sie erfüllt zwei
Hauptaufgaben:
\begin{enumerate}
\sphinxsetlistlabels{\arabic}{enumi}{enumii}{}{.}%
\item {} 
\sphinxAtStartPar
\sphinxstylestrong{Datei\sphinxhyphen{}Logging:} Schreibt alle Logs (DEBUG, INFO, WARNING, ERROR)
in eine zeitgestempelte .log\sphinxhyphen{}Datei im Benutzerverzeichnis
(\sphinxtitleref{\textasciitilde{}/Modulab/Logs}) unter Verwendung des standard \sphinxtitleref{logging}\sphinxhyphen{}Moduls.

\item {} 
\sphinxAtStartPar
\sphinxstylestrong{GUI\sphinxhyphen{}Benachrichtigung:} Speichert Logs im Speicher (in\sphinxhyphen{}memory) und
löst das \sphinxtitleref{message\_logged}\sphinxhyphen{}Signal für jeden neuen Eintrag aus,
um UIs (wie ein Log\sphinxhyphen{}Widget) in Echtzeit zu aktualisieren.

\end{enumerate}

\sphinxAtStartPar
Sie wird typischerweise einmal erstellt und an alle anderen Manager
übergeben.
\begin{description}
\sphinxlineitem{Signale:}\begin{description}
\sphinxlineitem{message\_logged (dict):}
\sphinxAtStartPar
Wird für jeden Log\sphinxhyphen{}Eintrag (Info, Error, etc.) ausgelöst.
Das übergebene Wörterbuch (dict) hat die Struktur:
\sphinxtitleref{\{‘timestamp’: datetime.datetime, ‘type’: str, ‘message’: str\}}

\end{description}

\end{description}
\index{INFO (modules.log.LogManager.LogManager attribute)@\spxentry{INFO}\spxextra{modules.log.LogManager.LogManager attribute}}

\begin{fulllineitems}
\phantomsection\label{\detokenize{modulab/log_manager:modules.log.LogManager.LogManager.INFO}}
\pysigstartsignatures
\pysigline
{\sphinxbfcode{\sphinxupquote{INFO}}\sphinxbfcode{\sphinxupquote{\DUrole{w}{ }\DUrole{p}{=}\DUrole{w}{ }\textquotesingle{}INFO\textquotesingle{}}}}
\pysigstopsignatures
\sphinxAtStartPar
Konstante für den ‘INFO’\sphinxhyphen{}Log\sphinxhyphen{}Level.
\begin{quote}\begin{description}
\sphinxlineitem{Type}
\sphinxAtStartPar
str

\end{description}\end{quote}

\end{fulllineitems}

\index{WARNING (modules.log.LogManager.LogManager attribute)@\spxentry{WARNING}\spxextra{modules.log.LogManager.LogManager attribute}}

\begin{fulllineitems}
\phantomsection\label{\detokenize{modulab/log_manager:modules.log.LogManager.LogManager.WARNING}}
\pysigstartsignatures
\pysigline
{\sphinxbfcode{\sphinxupquote{WARNING}}\sphinxbfcode{\sphinxupquote{\DUrole{w}{ }\DUrole{p}{=}\DUrole{w}{ }\textquotesingle{}WARNING\textquotesingle{}}}}
\pysigstopsignatures
\sphinxAtStartPar
Konstante für den ‘WARNING’\sphinxhyphen{}Log\sphinxhyphen{}Level.
\begin{quote}\begin{description}
\sphinxlineitem{Type}
\sphinxAtStartPar
str

\end{description}\end{quote}

\end{fulllineitems}

\index{ERROR (modules.log.LogManager.LogManager attribute)@\spxentry{ERROR}\spxextra{modules.log.LogManager.LogManager attribute}}

\begin{fulllineitems}
\phantomsection\label{\detokenize{modulab/log_manager:modules.log.LogManager.LogManager.ERROR}}
\pysigstartsignatures
\pysigline
{\sphinxbfcode{\sphinxupquote{ERROR}}\sphinxbfcode{\sphinxupquote{\DUrole{w}{ }\DUrole{p}{=}\DUrole{w}{ }\textquotesingle{}ERROR\textquotesingle{}}}}
\pysigstopsignatures
\sphinxAtStartPar
Konstante für den ‘ERROR’\sphinxhyphen{}Log\sphinxhyphen{}Level.
\begin{quote}\begin{description}
\sphinxlineitem{Type}
\sphinxAtStartPar
str

\end{description}\end{quote}

\end{fulllineitems}

\index{DEBUG (modules.log.LogManager.LogManager attribute)@\spxentry{DEBUG}\spxextra{modules.log.LogManager.LogManager attribute}}

\begin{fulllineitems}
\phantomsection\label{\detokenize{modulab/log_manager:modules.log.LogManager.LogManager.DEBUG}}
\pysigstartsignatures
\pysigline
{\sphinxbfcode{\sphinxupquote{DEBUG}}\sphinxbfcode{\sphinxupquote{\DUrole{w}{ }\DUrole{p}{=}\DUrole{w}{ }\textquotesingle{}DEBUG\textquotesingle{}}}}
\pysigstopsignatures
\sphinxAtStartPar
Konstante für den ‘DEBUG’\sphinxhyphen{}Log\sphinxhyphen{}Level.
\begin{quote}\begin{description}
\sphinxlineitem{Type}
\sphinxAtStartPar
str

\end{description}\end{quote}

\end{fulllineitems}

\index{\_\_init\_\_() (modules.log.LogManager.LogManager method)@\spxentry{\_\_init\_\_()}\spxextra{modules.log.LogManager.LogManager method}}

\begin{fulllineitems}
\phantomsection\label{\detokenize{modulab/log_manager:modules.log.LogManager.LogManager.__init__}}
\pysigstartsignatures
\pysiglinewithargsret
{\sphinxbfcode{\sphinxupquote{\_\_init\_\_}}}
{}
{}
\pysigstopsignatures
\sphinxAtStartPar
Initialisiert den LogManager.

\sphinxAtStartPar
Erstellt das Log\sphinxhyphen{}Verzeichnis (\sphinxtitleref{\textasciitilde{}/Modulab/Logs}), falls es nicht existiert,
und konfiguriert den Python \sphinxtitleref{logging}\sphinxhyphen{}Handler für das
zeitgestempelte Sitzungs\sphinxhyphen{}Logfile.
\begin{quote}\begin{description}
\sphinxlineitem{Raises}
\sphinxAtStartPar
\sphinxstyleliteralstrong{\sphinxupquote{RuntimeError}} \textendash{} Wenn das Log\sphinxhyphen{}Verzeichnis nicht erstellt oder
    beschrieben werden kann.

\end{description}\end{quote}

\end{fulllineitems}

\index{info() (modules.log.LogManager.LogManager method)@\spxentry{info()}\spxextra{modules.log.LogManager.LogManager method}}

\begin{fulllineitems}
\phantomsection\label{\detokenize{modulab/log_manager:modules.log.LogManager.LogManager.info}}
\pysigstartsignatures
\pysiglinewithargsret
{\sphinxbfcode{\sphinxupquote{info}}}
{\sphinxparam{\DUrole{n}{message}}}
{}
\pysigstopsignatures
\sphinxAtStartPar
Loggt eine Info\sphinxhyphen{}Nachricht.
\begin{quote}\begin{description}
\sphinxlineitem{Parameters}
\sphinxAtStartPar
\sphinxstyleliteralstrong{\sphinxupquote{message}} (\sphinxcode{\sphinxupquote{str}}) \textendash{} Die zu loggende Nachricht.

\end{description}\end{quote}

\begin{sphinxadmonition}{note}{Examples}

\sphinxAtStartPar
Eine einfache Statusmeldung loggen:

\begin{sphinxVerbatim}[commandchars=\\\{\}]
\PYG{c+c1}{\PYGZsh{} Annahme: \PYGZsq{}log\PYGZus{}mgr\PYGZsq{} ist eine Instanz von LogManager}
\PYG{n}{log\PYGZus{}mgr}\PYG{o}{.}\PYG{n}{info}\PYG{p}{(}\PYG{l+s+s2}{\PYGZdq{}}\PYG{l+s+s2}{Verbindung zum Gerät erfolgreich hergestellt.}\PYG{l+s+s2}{\PYGZdq{}}\PYG{p}{)}
\end{sphinxVerbatim}
\end{sphinxadmonition}

\end{fulllineitems}

\index{warning() (modules.log.LogManager.LogManager method)@\spxentry{warning()}\spxextra{modules.log.LogManager.LogManager method}}

\begin{fulllineitems}
\phantomsection\label{\detokenize{modulab/log_manager:modules.log.LogManager.LogManager.warning}}
\pysigstartsignatures
\pysiglinewithargsret
{\sphinxbfcode{\sphinxupquote{warning}}}
{\sphinxparam{\DUrole{n}{message}}}
{}
\pysigstopsignatures
\sphinxAtStartPar
Loggt eine Warnung.
\begin{quote}\begin{description}
\sphinxlineitem{Parameters}
\sphinxAtStartPar
\sphinxstyleliteralstrong{\sphinxupquote{message}} (\sphinxcode{\sphinxupquote{str}}) \textendash{} Die zu loggende Warnung.

\end{description}\end{quote}

\begin{sphinxadmonition}{note}{Examples}

\sphinxAtStartPar
Eine Warnung für einen ungültigen Wert loggen:

\begin{sphinxVerbatim}[commandchars=\\\{\}]
\PYG{n}{log\PYGZus{}mgr}\PYG{o}{.}\PYG{n}{warning}\PYG{p}{(}\PYG{l+s+s2}{\PYGZdq{}}\PYG{l+s+s2}{Integrationszeit auf 0 gesetzt. Ignoriere Befehl.}\PYG{l+s+s2}{\PYGZdq{}}\PYG{p}{)}
\end{sphinxVerbatim}
\end{sphinxadmonition}

\end{fulllineitems}

\index{error() (modules.log.LogManager.LogManager method)@\spxentry{error()}\spxextra{modules.log.LogManager.LogManager method}}

\begin{fulllineitems}
\phantomsection\label{\detokenize{modulab/log_manager:modules.log.LogManager.LogManager.error}}
\pysigstartsignatures
\pysiglinewithargsret
{\sphinxbfcode{\sphinxupquote{error}}}
{\sphinxparam{\DUrole{n}{message}}\sphinxparamcomma \sphinxparam{\DUrole{n}{exc\_info}\DUrole{o}{=}\DUrole{default_value}{True}}}
{}
\pysigstopsignatures
\sphinxAtStartPar
Loggt einen Fehler.

\sphinxAtStartPar
Standardmäßig wird die Ausnahme\sphinxhyphen{}Info (Stack Trace) mitgeloggt,
wenn diese Funktion innerhalb eines \sphinxtitleref{except}\sphinxhyphen{}Blocks aufgerufen wird.
\begin{quote}\begin{description}
\sphinxlineitem{Parameters}\begin{itemize}
\item {} 
\sphinxAtStartPar
\sphinxstyleliteralstrong{\sphinxupquote{message}} (\sphinxcode{\sphinxupquote{str}}) \textendash{} Die zu loggende Fehlermeldung.

\item {} 
\sphinxAtStartPar
\sphinxstyleliteralstrong{\sphinxupquote{exc\_info}} (\sphinxcode{\sphinxupquote{bool}}, \sphinxstyleemphasis{optional}) \textendash{} Wenn True (Standard), wird der
Stack Trace automatisch mitgeloggt.
Setzen Sie auf False, um dies zu unterdrücken.

\end{itemize}

\end{description}\end{quote}

\begin{sphinxadmonition}{note}{Examples}

\sphinxAtStartPar
Einen einfachen Fehler loggen (ohne Stack Trace):

\begin{sphinxVerbatim}[commandchars=\\\{\}]
\PYG{k}{if} \PYG{o+ow}{not} \PYG{n}{manager}\PYG{o}{.}\PYG{n}{is\PYGZus{}connected}\PYG{p}{(}\PYG{p}{)}\PYG{p}{:}
    \PYG{n}{log\PYGZus{}mgr}\PYG{o}{.}\PYG{n}{error}\PYG{p}{(}\PYG{l+s+s2}{\PYGZdq{}}\PYG{l+s+s2}{Verbindung fehlgeschlagen. Kein Stack Trace.}\PYG{l+s+s2}{\PYGZdq{}}\PYG{p}{,} \PYG{n}{exc\PYGZus{}info}\PYG{o}{=}\PYG{k+kc}{False}\PYG{p}{)}
\end{sphinxVerbatim}

\sphinxAtStartPar
Einen Fehler innerhalb einer Ausnahmebehandlung loggen (mit Stack Trace):

\begin{sphinxVerbatim}[commandchars=\\\{\}]
\PYG{k}{try}\PYG{p}{:}
    \PYG{c+c1}{\PYGZsh{} Code, der fehlschlagen könnte}
    \PYG{n}{result} \PYG{o}{=} \PYG{l+m+mi}{10} \PYG{o}{/} \PYG{l+m+mi}{0}
\PYG{k}{except} \PYG{n+ne}{ZeroDivisionError} \PYG{k}{as} \PYG{n}{e}\PYG{p}{:}
    \PYG{c+c1}{\PYGZsh{} \PYGZsq{}exc\PYGZus{}info=True\PYGZsq{} ist Standard, aber hier zur Verdeutlichung.}
    \PYG{c+c1}{\PYGZsh{} \PYGZsq{}e\PYGZsq{} wird automatisch vom Logger erfasst.}
    \PYG{n}{log\PYGZus{}mgr}\PYG{o}{.}\PYG{n}{error}\PYG{p}{(}\PYG{l+s+sa}{f}\PYG{l+s+s2}{\PYGZdq{}}\PYG{l+s+s2}{Schwerer Rechenfehler: }\PYG{l+s+si}{\PYGZob{}}\PYG{n}{e}\PYG{l+s+si}{\PYGZcb{}}\PYG{l+s+s2}{\PYGZdq{}}\PYG{p}{,} \PYG{n}{exc\PYGZus{}info}\PYG{o}{=}\PYG{k+kc}{True}\PYG{p}{)}
\end{sphinxVerbatim}
\end{sphinxadmonition}

\end{fulllineitems}

\index{debug() (modules.log.LogManager.LogManager method)@\spxentry{debug()}\spxextra{modules.log.LogManager.LogManager method}}

\begin{fulllineitems}
\phantomsection\label{\detokenize{modulab/log_manager:modules.log.LogManager.LogManager.debug}}
\pysigstartsignatures
\pysiglinewithargsret
{\sphinxbfcode{\sphinxupquote{debug}}}
{\sphinxparam{\DUrole{n}{message}}}
{}
\pysigstopsignatures
\sphinxAtStartPar
Loggt eine Debug\sphinxhyphen{}Nachricht.

\sphinxAtStartPar
Diese Nachrichten sind oft sehr detailliert und nur für die
Fehlersuche gedacht.
\begin{quote}\begin{description}
\sphinxlineitem{Parameters}
\sphinxAtStartPar
\sphinxstyleliteralstrong{\sphinxupquote{message}} (\sphinxcode{\sphinxupquote{str}}) \textendash{} Die zu loggende Debug\sphinxhyphen{}Nachricht.

\end{description}\end{quote}

\begin{sphinxadmonition}{note}{Examples}

\sphinxAtStartPar
Messwerte oder Zwischenschritte loggen:

\begin{sphinxVerbatim}[commandchars=\\\{\}]
\PYG{n}{log\PYGZus{}mgr}\PYG{o}{.}\PYG{n}{debug}\PYG{p}{(}\PYG{l+s+sa}{f}\PYG{l+s+s2}{\PYGZdq{}}\PYG{l+s+s2}{Spektrum aufgenommen. }\PYG{l+s+si}{\PYGZob{}}\PYG{n+nb}{len}\PYG{p}{(}\PYG{n}{intensities}\PYG{p}{)}\PYG{l+s+si}{\PYGZcb{}}\PYG{l+s+s2}{ Datenpunkte.}\PYG{l+s+s2}{\PYGZdq{}}\PYG{p}{)}
\end{sphinxVerbatim}
\end{sphinxadmonition}

\end{fulllineitems}

\index{get\_all\_messages() (modules.log.LogManager.LogManager method)@\spxentry{get\_all\_messages()}\spxextra{modules.log.LogManager.LogManager method}}

\begin{fulllineitems}
\phantomsection\label{\detokenize{modulab/log_manager:modules.log.LogManager.LogManager.get_all_messages}}
\pysigstartsignatures
\pysiglinewithargsret
{\sphinxbfcode{\sphinxupquote{get\_all\_messages}}}
{}
{}
\pysigstopsignatures
\sphinxAtStartPar
Gibt die komplette Liste aller Log\sphinxhyphen{}Einträge der aktuellen Sitzung zurück.

\sphinxAtStartPar
Nützlich, um ein Log\sphinxhyphen{}Fenster beim Öffnen zu initialisieren.
\begin{quote}\begin{description}
\sphinxlineitem{Returns}
\sphinxAtStartPar
\begin{description}
\sphinxlineitem{Eine Liste von Log\sphinxhyphen{}Einträgen. Jedes dict hat die}
\sphinxAtStartPar
Struktur: \sphinxtitleref{\{‘timestamp’: datetime, ‘type’: str, ‘message’: str\}}

\end{description}


\sphinxlineitem{Return type}
\sphinxAtStartPar
list{[}dict{]}

\end{description}\end{quote}

\begin{sphinxadmonition}{note}{Examples}

\sphinxAtStartPar
Alle bisherigen Logs beim Start eines Widgets laden:

\begin{sphinxVerbatim}[commandchars=\\\{\}]
\PYG{n}{alle\PYGZus{}logs} \PYG{o}{=} \PYG{n}{log\PYGZus{}gmr}\PYG{o}{.}\PYG{n}{get\PYGZus{}all\PYGZus{}messages}\PYG{p}{(}\PYG{p}{)}
\PYG{k}{for} \PYG{n}{eintrag} \PYG{o+ow}{in} \PYG{n}{alle\PYGZus{}logs}\PYG{p}{:}
    \PYG{c+c1}{\PYGZsh{} z.B. in eine QListWidget einfügen}
    \PYG{n+nb}{print}\PYG{p}{(}\PYG{l+s+sa}{f}\PYG{l+s+s2}{\PYGZdq{}}\PYG{l+s+s2}{[}\PYG{l+s+si}{\PYGZob{}}\PYG{n}{eintrag}\PYG{p}{[}\PYG{l+s+s1}{\PYGZsq{}}\PYG{l+s+s1}{type}\PYG{l+s+s1}{\PYGZsq{}}\PYG{p}{]}\PYG{l+s+si}{\PYGZcb{}}\PYG{l+s+s2}{] }\PYG{l+s+si}{\PYGZob{}}\PYG{n}{eintrag}\PYG{p}{[}\PYG{l+s+s1}{\PYGZsq{}}\PYG{l+s+s1}{message}\PYG{l+s+s1}{\PYGZsq{}}\PYG{p}{]}\PYG{l+s+si}{\PYGZcb{}}\PYG{l+s+s2}{\PYGZdq{}}\PYG{p}{)}
\end{sphinxVerbatim}
\end{sphinxadmonition}

\end{fulllineitems}

\index{get\_latest\_message() (modules.log.LogManager.LogManager method)@\spxentry{get\_latest\_message()}\spxextra{modules.log.LogManager.LogManager method}}

\begin{fulllineitems}
\phantomsection\label{\detokenize{modulab/log_manager:modules.log.LogManager.LogManager.get_latest_message}}
\pysigstartsignatures
\pysiglinewithargsret
{\sphinxbfcode{\sphinxupquote{get\_latest\_message}}}
{}
{}
\pysigstopsignatures
\sphinxAtStartPar
Gibt nur den letzten Log\sphinxhyphen{}Eintrag zurück. (Für bspw. Status Label)
\begin{quote}\begin{description}
\sphinxlineitem{Returns}
\sphinxAtStartPar
\begin{description}
\sphinxlineitem{Der letzte Log\sphinxhyphen{}Eintrag als dict mit der Struktur}
\sphinxAtStartPar
\sphinxtitleref{\{‘timestamp’: …, ‘type’: …, ‘message’: …\}}
oder \sphinxtitleref{None}, wenn noch keine Logs vorhanden sind.

\end{description}


\sphinxlineitem{Return type}
\sphinxAtStartPar
dict | None

\end{description}\end{quote}

\begin{sphinxadmonition}{note}{Examples}

\sphinxAtStartPar
Den Text für eine Statusleiste setzen:

\begin{sphinxVerbatim}[commandchars=\\\{\}]
\PYG{n}{letzte\PYGZus{}meldung} \PYG{o}{=} \PYG{n}{log\PYGZus{}mgr}\PYG{o}{.}\PYG{n}{get\PYGZus{}latest\PYGZus{}message}\PYG{p}{(}\PYG{p}{)}
\PYG{k}{if} \PYG{n}{letzte\PYGZus{}meldung}\PYG{p}{:}
    \PYG{n}{status\PYGZus{}bar}\PYG{o}{.}\PYG{n}{showMessage}\PYG{p}{(}\PYG{n}{letzte\PYGZus{}meldung}\PYG{p}{[}\PYG{l+s+s1}{\PYGZsq{}}\PYG{l+s+s1}{message}\PYG{l+s+s1}{\PYGZsq{}}\PYG{p}{]}\PYG{p}{)}
\end{sphinxVerbatim}
\end{sphinxadmonition}

\end{fulllineitems}


\end{fulllineitems}


\sphinxstepscope


\chapter{Profile Manager}
\label{\detokenize{modulab/profile_manager:module-modules.profile.ProfileManager}}\label{\detokenize{modulab/profile_manager:id1}}\label{\detokenize{modulab/profile_manager:profile-manager}}\label{\detokenize{modulab/profile_manager::doc}}\index{module@\spxentry{module}!modules.profile.ProfileManager@\spxentry{modules.profile.ProfileManager}}\index{modules.profile.ProfileManager@\spxentry{modules.profile.ProfileManager}!module@\spxentry{module}}\index{ProfileManager (class in modules.profile.ProfileManager)@\spxentry{ProfileManager}\spxextra{class in modules.profile.ProfileManager}}

\begin{fulllineitems}
\phantomsection\label{\detokenize{modulab/profile_manager:modules.profile.ProfileManager.ProfileManager}}
\pysigstartsignatures
\pysiglinewithargsret
{\sphinxbfcode{\sphinxupquote{\DUrole{k}{class}\DUrole{w}{ }}}\sphinxcode{\sphinxupquote{modules.profile.ProfileManager.}}\sphinxbfcode{\sphinxupquote{ProfileManager}}}
{\sphinxparam{\DUrole{n}{log\_manager}}}
{}
\pysigstopsignatures
\sphinxAtStartPar
Bases: \sphinxcode{\sphinxupquote{QObject}}

\sphinxAtStartPar
Erstellt, speichert und verwaltet Benutzerprofile (Key\sphinxhyphen{}Value\sphinxhyphen{}Speicher).

\sphinxAtStartPar
Diese Klasse verwaltet app\sphinxhyphen{}weite Einstellungen, indem sie diese in
\sphinxtitleref{.json}\sphinxhyphen{}Dateien im Benutzerverzeichnis (\sphinxtitleref{\textasciitilde{}/Modulab/Profiles}) speichert.
Jede \sphinxtitleref{.json}\sphinxhyphen{}Datei repräsentiert ein “Profil” (z.B. “Experiment\_A”, “Default”).

\sphinxAtStartPar
Sie dient als zentraler Key\sphinxhyphen{}Value\sphinxhyphen{}Speicher für alle anderen Manager
(z.B. zum Speichern der letzten Integrationszeit oder des letzten
verbundenen Geräts).
\begin{quote}\begin{description}
\sphinxlineitem{Parameters}
\sphinxAtStartPar
\sphinxstyleliteralstrong{\sphinxupquote{log\_manager}} (\sphinxcode{\sphinxupquote{LogManager}}) \textendash{} Eine Instanz des LogManagers für das Logging.

\end{description}\end{quote}
\begin{description}
\sphinxlineitem{Signale:}\begin{description}
\sphinxlineitem{profile\_loaded (str):}
\sphinxAtStartPar
Wird ausgelöst, nachdem \sphinxtitleref{load\_profile} erfolgreich war.
Args: (str: Der Name des geladenen Profils).

\end{description}

\end{description}
\index{CONFIG\_FILE\_NAME (modules.profile.ProfileManager.ProfileManager attribute)@\spxentry{CONFIG\_FILE\_NAME}\spxextra{modules.profile.ProfileManager.ProfileManager attribute}}

\begin{fulllineitems}
\phantomsection\label{\detokenize{modulab/profile_manager:modules.profile.ProfileManager.ProfileManager.CONFIG_FILE_NAME}}
\pysigstartsignatures
\pysigline
{\sphinxbfcode{\sphinxupquote{CONFIG\_FILE\_NAME}}\sphinxbfcode{\sphinxupquote{\DUrole{w}{ }\DUrole{p}{=}\DUrole{w}{ }\textquotesingle{}config.json\textquotesingle{}}}}
\pysigstopsignatures
\sphinxAtStartPar
Interner Dateiname für die Manager\sphinxhyphen{}Konfiguration (z.B. letztes Profil).
\begin{quote}\begin{description}
\sphinxlineitem{Type}
\sphinxAtStartPar
str

\end{description}\end{quote}

\end{fulllineitems}

\index{KEY\_LAST\_PROFILE (modules.profile.ProfileManager.ProfileManager attribute)@\spxentry{KEY\_LAST\_PROFILE}\spxextra{modules.profile.ProfileManager.ProfileManager attribute}}

\begin{fulllineitems}
\phantomsection\label{\detokenize{modulab/profile_manager:modules.profile.ProfileManager.ProfileManager.KEY_LAST_PROFILE}}
\pysigstartsignatures
\pysigline
{\sphinxbfcode{\sphinxupquote{KEY\_LAST\_PROFILE}}\sphinxbfcode{\sphinxupquote{\DUrole{w}{ }\DUrole{p}{=}\DUrole{w}{ }\textquotesingle{}last\_profile\_name\textquotesingle{}}}}
\pysigstopsignatures
\sphinxAtStartPar
Interner Schlüssel zum Speichern des Namens des zuletzt geladenen Profils.
\begin{quote}\begin{description}
\sphinxlineitem{Type}
\sphinxAtStartPar
str

\end{description}\end{quote}

\end{fulllineitems}

\index{\_\_init\_\_() (modules.profile.ProfileManager.ProfileManager method)@\spxentry{\_\_init\_\_()}\spxextra{modules.profile.ProfileManager.ProfileManager method}}

\begin{fulllineitems}
\phantomsection\label{\detokenize{modulab/profile_manager:modules.profile.ProfileManager.ProfileManager.__init__}}
\pysigstartsignatures
\pysiglinewithargsret
{\sphinxbfcode{\sphinxupquote{\_\_init\_\_}}}
{\sphinxparam{\DUrole{n}{log\_manager}}}
{}
\pysigstopsignatures
\sphinxAtStartPar
Initialisiert den ProfileManager.

\sphinxAtStartPar
Erstellt das Profil\sphinxhyphen{}Verzeichnis (\sphinxtitleref{\textasciitilde{}/Modulab/Profiles}), falls es
nicht existiert.
\begin{quote}\begin{description}
\sphinxlineitem{Parameters}
\sphinxAtStartPar
\sphinxstyleliteralstrong{\sphinxupquote{log\_manager}} (\sphinxcode{\sphinxupquote{LogManager}}) \textendash{} Die LogManager\sphinxhyphen{}Instanz.

\end{description}\end{quote}

\end{fulllineitems}

\index{create\_profile() (modules.profile.ProfileManager.ProfileManager method)@\spxentry{create\_profile()}\spxextra{modules.profile.ProfileManager.ProfileManager method}}

\begin{fulllineitems}
\phantomsection\label{\detokenize{modulab/profile_manager:modules.profile.ProfileManager.ProfileManager.create_profile}}
\pysigstartsignatures
\pysiglinewithargsret
{\sphinxbfcode{\sphinxupquote{create\_profile}}}
{\sphinxparam{\DUrole{n}{profile\_name}}\sphinxparamcomma \sphinxparam{\DUrole{n}{data}\DUrole{o}{=}\DUrole{default_value}{None}}}
{}
\pysigstopsignatures
\sphinxAtStartPar
Erstellt eine neue, leere Profildatei (.json).

\sphinxAtStartPar
Wenn \sphinxtitleref{data} angegeben wird, wird das Profil mit diesen
Anfangswerten erstellt.
\begin{quote}\begin{description}
\sphinxlineitem{Parameters}\begin{itemize}
\item {} 
\sphinxAtStartPar
\sphinxstyleliteralstrong{\sphinxupquote{profile\_name}} (\sphinxcode{\sphinxupquote{str}}) \textendash{} Der Name für das neue Profil (ohne .json).

\item {} 
\sphinxAtStartPar
\sphinxstyleliteralstrong{\sphinxupquote{data}} (\sphinxcode{\sphinxupquote{dict}}, \sphinxstyleemphasis{optional}) \textendash{} Ein Wörterbuch mit Anfangsdaten
für das Profil.

\end{itemize}

\sphinxlineitem{Returns}
\sphinxAtStartPar
\begin{description}
\sphinxlineitem{True bei Erfolg, False, wenn das Profil bereits existiert}
\sphinxAtStartPar
oder ein Fehler auftritt.

\end{description}


\sphinxlineitem{Return type}
\sphinxAtStartPar
bool

\end{description}\end{quote}

\begin{sphinxadmonition}{note}{Examples}

\sphinxAtStartPar
Ein neues, leeres Profil “Experiment\_A” erstellen:

\begin{sphinxVerbatim}[commandchars=\\\{\}]
\PYG{n}{profile\PYGZus{}mgr}\PYG{o}{.}\PYG{n}{create\PYGZus{}profile}\PYG{p}{(}\PYG{l+s+s2}{\PYGZdq{}}\PYG{l+s+s2}{Experiment\PYGZus{}A}\PYG{l+s+s2}{\PYGZdq{}}\PYG{p}{)}
\end{sphinxVerbatim}

\sphinxAtStartPar
Ein Profil mit Standard\sphinxhyphen{}Einstellungen für ein Spektrometer erstellen:

\begin{sphinxVerbatim}[commandchars=\\\{\}]
\PYG{n}{default\PYGZus{}settings} \PYG{o}{=} \PYG{p}{\PYGZob{}}
    \PYG{l+s+s2}{\PYGZdq{}}\PYG{l+s+s2}{Spec\PYGZus{}integration\PYGZus{}time\PYGZus{}us}\PYG{l+s+s2}{\PYGZdq{}}\PYG{p}{:} \PYG{l+m+mi}{10000}\PYG{p}{,}
    \PYG{l+s+s2}{\PYGZdq{}}\PYG{l+s+s2}{Spec\PYGZus{}correct\PYGZus{}dark\PYGZus{}counts}\PYG{l+s+s2}{\PYGZdq{}}\PYG{p}{:} \PYG{k+kc}{False}
\PYG{p}{\PYGZcb{}}
\PYG{n}{profile\PYGZus{}mgr}\PYG{o}{.}\PYG{n}{create\PYGZus{}profile}\PYG{p}{(}\PYG{l+s+s2}{\PYGZdq{}}\PYG{l+s+s2}{Default\PYGZus{}Spektrometer}\PYG{l+s+s2}{\PYGZdq{}}\PYG{p}{,} \PYG{n}{data}\PYG{o}{=}\PYG{n}{default\PYGZus{}settings}\PYG{p}{)}
\end{sphinxVerbatim}
\end{sphinxadmonition}

\end{fulllineitems}

\index{load\_profile() (modules.profile.ProfileManager.ProfileManager method)@\spxentry{load\_profile()}\spxextra{modules.profile.ProfileManager.ProfileManager method}}

\begin{fulllineitems}
\phantomsection\label{\detokenize{modulab/profile_manager:modules.profile.ProfileManager.ProfileManager.load_profile}}
\pysigstartsignatures
\pysiglinewithargsret
{\sphinxbfcode{\sphinxupquote{load\_profile}}}
{\sphinxparam{\DUrole{n}{profile\_name}}}
{}
\pysigstopsignatures
\sphinxAtStartPar
Lädt ein Profil in den Speicher und macht es zum “aktuellen” Profil.

\sphinxAtStartPar
Alle zukünftigen \sphinxtitleref{read()}\sphinxhyphen{} und \sphinxtitleref{write()}\sphinxhyphen{}Operationen beziehen sich
auf dieses Profil. Setzt auch dieses Profil als “zuletzt verwendet”.
\begin{quote}\begin{description}
\sphinxlineitem{Parameters}
\sphinxAtStartPar
\sphinxstyleliteralstrong{\sphinxupquote{profile\_name}} (\sphinxcode{\sphinxupquote{str}}) \textendash{} Der Name des zu ladenden Profils.

\sphinxlineitem{Returns}
\sphinxAtStartPar
True bei Erfolg, False, wenn das Profil nicht gefunden wurde.

\sphinxlineitem{Return type}
\sphinxAtStartPar
bool

\end{description}\end{quote}

\begin{sphinxadmonition}{note}{Examples}

\sphinxAtStartPar
Das “Default” Profil beim Start laden:

\begin{sphinxVerbatim}[commandchars=\\\{\}]
\PYG{k}{if} \PYG{o+ow}{not} \PYG{n}{profile\PYGZus{}mgr}\PYG{o}{.}\PYG{n}{load\PYGZus{}profile}\PYG{p}{(}\PYG{l+s+s2}{\PYGZdq{}}\PYG{l+s+s2}{Default}\PYG{l+s+s2}{\PYGZdq{}}\PYG{p}{)}\PYG{p}{:}
    \PYG{n}{profile\PYGZus{}mgr}\PYG{o}{.}\PYG{n}{create\PYGZus{}profile}\PYG{p}{(}\PYG{l+s+s2}{\PYGZdq{}}\PYG{l+s+s2}{Default}\PYG{l+s+s2}{\PYGZdq{}}\PYG{p}{)}
    \PYG{n}{profile\PYGZus{}mgr}\PYG{o}{.}\PYG{n}{load\PYGZus{}profile}\PYG{p}{(}\PYG{l+s+s2}{\PYGZdq{}}\PYG{l+s+s2}{Default}\PYG{l+s+s2}{\PYGZdq{}}\PYG{p}{)}
\end{sphinxVerbatim}
\end{sphinxadmonition}

\end{fulllineitems}

\index{delete\_profile() (modules.profile.ProfileManager.ProfileManager method)@\spxentry{delete\_profile()}\spxextra{modules.profile.ProfileManager.ProfileManager method}}

\begin{fulllineitems}
\phantomsection\label{\detokenize{modulab/profile_manager:modules.profile.ProfileManager.ProfileManager.delete_profile}}
\pysigstartsignatures
\pysiglinewithargsret
{\sphinxbfcode{\sphinxupquote{delete\_profile}}}
{\sphinxparam{\DUrole{n}{profile\_name}}}
{}
\pysigstopsignatures
\sphinxAtStartPar
Löscht eine Profildatei (.json) vom Datenträger.
\begin{quote}\begin{description}
\sphinxlineitem{Parameters}
\sphinxAtStartPar
\sphinxstyleliteralstrong{\sphinxupquote{profile\_name}} (\sphinxcode{\sphinxupquote{str}}) \textendash{} Der Name des zu löschenden Profils.

\sphinxlineitem{Returns}
\sphinxAtStartPar
\begin{description}
\sphinxlineitem{True bei Erfolg, False, wenn die Datei nicht existiert}
\sphinxAtStartPar
oder ein Fehler auftritt.

\end{description}


\sphinxlineitem{Return type}
\sphinxAtStartPar
bool

\end{description}\end{quote}

\end{fulllineitems}

\index{list\_profiles() (modules.profile.ProfileManager.ProfileManager method)@\spxentry{list\_profiles()}\spxextra{modules.profile.ProfileManager.ProfileManager method}}

\begin{fulllineitems}
\phantomsection\label{\detokenize{modulab/profile_manager:modules.profile.ProfileManager.ProfileManager.list_profiles}}
\pysigstartsignatures
\pysiglinewithargsret
{\sphinxbfcode{\sphinxupquote{list\_profiles}}}
{}
{}
\pysigstopsignatures
\sphinxAtStartPar
Listet alle verfügbaren Profile (alle .json\sphinxhyphen{}Dateien) im Profilordner auf.
\begin{quote}\begin{description}
\sphinxlineitem{Returns}
\sphinxAtStartPar
Eine alphabetisch sortierte Liste aller Profilnamen.

\sphinxlineitem{Return type}
\sphinxAtStartPar
list{[}str{]}

\end{description}\end{quote}

\begin{sphinxadmonition}{note}{Examples}

\sphinxAtStartPar
Eine QComboBox mit allen Profilen füllen:

\begin{sphinxVerbatim}[commandchars=\\\{\}]
\PYG{n}{profil\PYGZus{}liste} \PYG{o}{=} \PYG{n}{profile\PYGZus{}mgr}\PYG{o}{.}\PYG{n}{list\PYGZus{}profiles}\PYG{p}{(}\PYG{p}{)}
\PYG{n}{ui}\PYG{o}{.}\PYG{n}{profile\PYGZus{}combobox}\PYG{o}{.}\PYG{n}{clear}\PYG{p}{(}\PYG{p}{)}
\PYG{n}{ui}\PYG{o}{.}\PYG{n}{profile\PYGZus{}combobox}\PYG{o}{.}\PYG{n}{addItems}\PYG{p}{(}\PYG{n}{profil\PYGZus{}liste}\PYG{p}{)}
\end{sphinxVerbatim}
\end{sphinxadmonition}

\end{fulllineitems}

\index{write() (modules.profile.ProfileManager.ProfileManager method)@\spxentry{write()}\spxextra{modules.profile.ProfileManager.ProfileManager method}}

\begin{fulllineitems}
\phantomsection\label{\detokenize{modulab/profile_manager:modules.profile.ProfileManager.ProfileManager.write}}
\pysigstartsignatures
\pysiglinewithargsret
{\sphinxbfcode{\sphinxupquote{write}}}
{\sphinxparam{\DUrole{n}{key}}\sphinxparamcomma \sphinxparam{\DUrole{n}{value}}}
{}
\pysigstopsignatures
\sphinxAtStartPar
Schreibt ein Key\sphinxhyphen{}Value\sphinxhyphen{}Paar in das \sphinxstyleemphasis{aktuell geladene} Profil.

\sphinxAtStartPar
Dies ist die Hauptmethode für andere Manager, um ihre
Einstellungen zu speichern.
\begin{quote}\begin{description}
\sphinxlineitem{Parameters}\begin{itemize}
\item {} 
\sphinxAtStartPar
\sphinxstyleliteralstrong{\sphinxupquote{key}} (\sphinxcode{\sphinxupquote{str}}) \textendash{} Der Einstellungs\sphinxhyphen{}Schlüssel (z.B. “Spec\_integration\_time\_us”).

\item {} 
\sphinxAtStartPar
\sphinxstyleliteralstrong{\sphinxupquote{value}} (\sphinxcode{\sphinxupquote{any}}) \textendash{} Der zugehörige Wert (muss JSON\sphinxhyphen{}serialisierbar sein).

\end{itemize}

\sphinxlineitem{Returns}
\sphinxAtStartPar
True bei Erfolg, False, wenn kein Profil geladen ist.

\sphinxlineitem{Return type}
\sphinxAtStartPar
bool

\end{description}\end{quote}

\begin{sphinxadmonition}{note}{Examples}

\sphinxAtStartPar
Einstellungen von anderen Managern speichern:

\begin{sphinxVerbatim}[commandchars=\\\{\}]
\PYG{c+c1}{\PYGZsh{} Im SpectrometerManager (nach Änderung der Integrationszeit)}
\PYG{n}{time\PYGZus{}us} \PYG{o}{=} \PYG{l+m+mi}{100000}
\PYG{n}{profile\PYGZus{}mgr}\PYG{o}{.}\PYG{n}{write}\PYG{p}{(}\PYG{l+s+s2}{\PYGZdq{}}\PYG{l+s+s2}{Spec\PYGZus{}integration\PYGZus{}time\PYGZus{}us}\PYG{l+s+s2}{\PYGZdq{}}\PYG{p}{,} \PYG{n}{time\PYGZus{}us}\PYG{p}{)}

\PYG{c+c1}{\PYGZsh{} Im SmuManager (nach Verbindung)}
\PYG{n}{port} \PYG{o}{=} \PYG{l+s+s2}{\PYGZdq{}}\PYG{l+s+s2}{COM3}\PYG{l+s+s2}{\PYGZdq{}}
\PYG{n}{profile\PYGZus{}mgr}\PYG{o}{.}\PYG{n}{write}\PYG{p}{(}\PYG{l+s+s2}{\PYGZdq{}}\PYG{l+s+s2}{Smu\PYGZus{}LastDevice}\PYG{l+s+s2}{\PYGZdq{}}\PYG{p}{,} \PYG{n}{port}\PYG{p}{)}
\end{sphinxVerbatim}
\end{sphinxadmonition}

\end{fulllineitems}

\index{read() (modules.profile.ProfileManager.ProfileManager method)@\spxentry{read()}\spxextra{modules.profile.ProfileManager.ProfileManager method}}

\begin{fulllineitems}
\phantomsection\label{\detokenize{modulab/profile_manager:modules.profile.ProfileManager.ProfileManager.read}}
\pysigstartsignatures
\pysiglinewithargsret
{\sphinxbfcode{\sphinxupquote{read}}}
{\sphinxparam{\DUrole{n}{key}}}
{}
\pysigstopsignatures
\sphinxAtStartPar
Liest einen Wert anhand des ‘key’ aus dem \sphinxstyleemphasis{aktuell geladenen} Profil.

\sphinxAtStartPar
Dies ist die Hauptmethode für andere Manager, um ihre
Einstellungen zu laden.
\begin{quote}\begin{description}
\sphinxlineitem{Parameters}
\sphinxAtStartPar
\sphinxstyleliteralstrong{\sphinxupquote{key}} (\sphinxcode{\sphinxupquote{str}}) \textendash{} Der Einstellungs\sphinxhyphen{}Schlüssel (z.B. “Spec\_integration\_time\_us”).

\sphinxlineitem{Returns}
\sphinxAtStartPar
\begin{description}
\sphinxlineitem{Der gespeicherte Wert, oder \sphinxtitleref{None}, wenn der Schlüssel}
\sphinxAtStartPar
nicht existiert oder kein Profil geladen ist.

\end{description}


\sphinxlineitem{Return type}
\sphinxAtStartPar
any | None

\end{description}\end{quote}

\begin{sphinxadmonition}{note}{Examples}

\sphinxAtStartPar
Einstellungen in anderen Managern laden:

\begin{sphinxVerbatim}[commandchars=\\\{\}]
\PYG{c+c1}{\PYGZsh{} Im SpectrometerManager (während \PYGZus{}\PYGZus{}init\PYGZus{}\PYGZus{})}
\PYG{c+c1}{\PYGZsh{} Lade gespeicherte Zeit, oder nutze 100ms als Standard}
\PYG{n}{default\PYGZus{}time} \PYG{o}{=} \PYG{l+m+mi}{100} \PYG{o}{*} \PYG{l+m+mi}{1000}
\PYG{n}{time\PYGZus{}us} \PYG{o}{=} \PYG{n}{profile\PYGZus{}mgr}\PYG{o}{.}\PYG{n}{read}\PYG{p}{(}\PYG{l+s+s2}{\PYGZdq{}}\PYG{l+s+s2}{Spec\PYGZus{}integration\PYGZus{}time\PYGZus{}us}\PYG{l+s+s2}{\PYGZdq{}}\PYG{p}{)}

\PYG{k}{if} \PYG{n}{time\PYGZus{}us} \PYG{o+ow}{is} \PYG{k+kc}{None}\PYG{p}{:}
    \PYG{n}{time\PYGZus{}us} \PYG{o}{=} \PYG{n}{default\PYGZus{}time}

\PYG{n+nb+bp}{self}\PYG{o}{.}\PYG{n}{set\PYGZus{}integrationtime}\PYG{p}{(}\PYG{n}{time\PYGZus{}us}\PYG{p}{)}
\end{sphinxVerbatim}
\end{sphinxadmonition}

\end{fulllineitems}

\index{get\_current\_profile\_name() (modules.profile.ProfileManager.ProfileManager method)@\spxentry{get\_current\_profile\_name()}\spxextra{modules.profile.ProfileManager.ProfileManager method}}

\begin{fulllineitems}
\phantomsection\label{\detokenize{modulab/profile_manager:modules.profile.ProfileManager.ProfileManager.get_current_profile_name}}
\pysigstartsignatures
\pysiglinewithargsret
{\sphinxbfcode{\sphinxupquote{get\_current\_profile\_name}}}
{}
{}
\pysigstopsignatures
\sphinxAtStartPar
Gibt den Namen des aktuell geladenen Profils zurück.
\begin{quote}\begin{description}
\sphinxlineitem{Returns}
\sphinxAtStartPar
Der Name des Profils, oder \sphinxtitleref{None}.

\sphinxlineitem{Return type}
\sphinxAtStartPar
str | None

\end{description}\end{quote}

\end{fulllineitems}

\index{get\_last\_profile\_name() (modules.profile.ProfileManager.ProfileManager method)@\spxentry{get\_last\_profile\_name()}\spxextra{modules.profile.ProfileManager.ProfileManager method}}

\begin{fulllineitems}
\phantomsection\label{\detokenize{modulab/profile_manager:modules.profile.ProfileManager.ProfileManager.get_last_profile_name}}
\pysigstartsignatures
\pysiglinewithargsret
{\sphinxbfcode{\sphinxupquote{get\_last\_profile\_name}}}
{}
{}
\pysigstopsignatures
\sphinxAtStartPar
Liest aus der \sphinxtitleref{config.json}, welches Profil zuletzt geladen wurde.

\sphinxAtStartPar
Prüft gleichzeitig, ob dieses Profil noch existiert.
\begin{quote}\begin{description}
\sphinxlineitem{Returns}
\sphinxAtStartPar
Der Name des letzten Profils, oder \sphinxtitleref{None}.

\sphinxlineitem{Return type}
\sphinxAtStartPar
str | None

\end{description}\end{quote}

\begin{sphinxadmonition}{note}{Examples}

\sphinxAtStartPar
Beim App\sphinxhyphen{}Start das letzte Profil automatisch laden:

\begin{sphinxVerbatim}[commandchars=\\\{\}]
\PYG{n}{last\PYGZus{}profile} \PYG{o}{=} \PYG{n}{profile\PYGZus{}mgr}\PYG{o}{.}\PYG{n}{get\PYGZus{}last\PYGZus{}profile\PYGZus{}name}\PYG{p}{(}\PYG{p}{)}
\PYG{k}{if} \PYG{n}{last\PYGZus{}profile}\PYG{p}{:}
    \PYG{n}{profile\PYGZus{}mgr}\PYG{o}{.}\PYG{n}{load\PYGZus{}profile}\PYG{p}{(}\PYG{n}{last\PYGZus{}profile}\PYG{p}{)}
\PYG{k}{else}\PYG{p}{:}
    \PYG{c+c1}{\PYGZsh{} Lade Fallback oder zeige Profil\PYGZhy{}Auswahl}
    \PYG{n}{profile\PYGZus{}mgr}\PYG{o}{.}\PYG{n}{load\PYGZus{}profile}\PYG{p}{(}\PYG{l+s+s2}{\PYGZdq{}}\PYG{l+s+s2}{Default}\PYG{l+s+s2}{\PYGZdq{}}\PYG{p}{)}
\end{sphinxVerbatim}
\end{sphinxadmonition}

\end{fulllineitems}


\end{fulllineitems}


\sphinxstepscope


\chapter{Device Manager}
\label{\detokenize{modulab/device_manager:module-modules.device.DeviceManager}}\label{\detokenize{modulab/device_manager:id1}}\label{\detokenize{modulab/device_manager:device-manager}}\label{\detokenize{modulab/device_manager::doc}}\index{module@\spxentry{module}!modules.device.DeviceManager@\spxentry{modules.device.DeviceManager}}\index{modules.device.DeviceManager@\spxentry{modules.device.DeviceManager}!module@\spxentry{module}}\index{Device (class in modules.device.DeviceManager)@\spxentry{Device}\spxextra{class in modules.device.DeviceManager}}

\begin{fulllineitems}
\phantomsection\label{\detokenize{modulab/device_manager:modules.device.DeviceManager.Device}}
\pysigstartsignatures
\pysiglinewithargsret
{\sphinxbfcode{\sphinxupquote{\DUrole{k}{class}\DUrole{w}{ }}}\sphinxcode{\sphinxupquote{modules.device.DeviceManager.}}\sphinxbfcode{\sphinxupquote{Device}}}
{\sphinxparam{\DUrole{n}{name}}\sphinxparamcomma \sphinxparam{\DUrole{n}{geometry}}\sphinxparamcomma \sphinxparam{\DUrole{n}{tags}\DUrole{o}{=}\DUrole{default_value}{None}}\sphinxparamcomma \sphinxparam{\DUrole{o}{**}\DUrole{n}{dimensions}}}
{}
\pysigstopsignatures
\sphinxAtStartPar
Bases: \sphinxcode{\sphinxupquote{object}}

\sphinxAtStartPar
Repräsentiert ein einzelnes physisches Device (z.B. Pixel, Teststruktur).

\sphinxAtStartPar
Dies ist eine Daten\sphinxhyphen{}Container\sphinxhyphen{}Klasse, die alle geometrischen und
metadatenbezogenen Eigenschaften eines Devices speichert.
\index{\_\_init\_\_() (modules.device.DeviceManager.Device method)@\spxentry{\_\_init\_\_()}\spxextra{modules.device.DeviceManager.Device method}}

\begin{fulllineitems}
\phantomsection\label{\detokenize{modulab/device_manager:modules.device.DeviceManager.Device.__init__}}
\pysigstartsignatures
\pysiglinewithargsret
{\sphinxbfcode{\sphinxupquote{\_\_init\_\_}}}
{\sphinxparam{\DUrole{n}{name}}\sphinxparamcomma \sphinxparam{\DUrole{n}{geometry}}\sphinxparamcomma \sphinxparam{\DUrole{n}{tags}\DUrole{o}{=}\DUrole{default_value}{None}}\sphinxparamcomma \sphinxparam{\DUrole{o}{**}\DUrole{n}{dimensions}}}
{}
\pysigstopsignatures
\sphinxAtStartPar
Initialisiert ein neues Device\sphinxhyphen{}Objekt.
\begin{quote}\begin{description}
\sphinxlineitem{Parameters}\begin{itemize}
\item {} 
\sphinxAtStartPar
\sphinxstyleliteralstrong{\sphinxupquote{name}} (\sphinxcode{\sphinxupquote{str}}) \textendash{} Der eindeutige Name des Devices (z.B. “Pixel\_R1C1”).

\item {} 
\sphinxAtStartPar
\sphinxstyleliteralstrong{\sphinxupquote{geometry}} (\sphinxcode{\sphinxupquote{str}}) \textendash{} Der Geometrie\sphinxhyphen{}Typ. Muss ‘rectangle’ oder ‘circle’ sein.

\item {} 
\sphinxAtStartPar
\sphinxstyleliteralstrong{\sphinxupquote{tags}} (\sphinxcode{\sphinxupquote{list}}, \sphinxstyleemphasis{optional}) \textendash{} Eine Liste von String\sphinxhyphen{}Tags zur Filterung.

\item {} 
\sphinxAtStartPar
\sphinxstyleliteralstrong{\sphinxupquote{**dimensions}} (\sphinxcode{\sphinxupquote{float}}) \textendash{} 
\sphinxAtStartPar
Dynamische Schlüsselwortargumente für die Maße
des Devices in Metern {[}m{]}.
\begin{itemize}
\item {} 
\sphinxAtStartPar
Für ‘rectangle’: \sphinxtitleref{length}, \sphinxtitleref{width}

\item {} 
\sphinxAtStartPar
Für ‘circle’: \sphinxtitleref{radius}

\item {} 
\sphinxAtStartPar
Optionale Cutouts: \sphinxtitleref{cutout\_length}, \sphinxtitleref{cutout\_width},
\sphinxtitleref{cutout\_radius}.

\end{itemize}


\end{itemize}

\end{description}\end{quote}

\begin{sphinxadmonition}{note}{Examples}

\sphinxAtStartPar
Ein rechteckiges Device (1mm x 0.5mm) erstellen:

\begin{sphinxVerbatim}[commandchars=\\\{\}]
\PYG{n}{dev1} \PYG{o}{=} \PYG{n}{Device}\PYG{p}{(}
    \PYG{n}{name}\PYG{o}{=}\PYG{l+s+s2}{\PYGZdq{}}\PYG{l+s+s2}{Pixel\PYGZus{}1}\PYG{l+s+s2}{\PYGZdq{}}\PYG{p}{,}
    \PYG{n}{geometry}\PYG{o}{=}\PYG{l+s+s2}{\PYGZdq{}}\PYG{l+s+s2}{rectangle}\PYG{l+s+s2}{\PYGZdq{}}\PYG{p}{,}
    \PYG{n}{tags}\PYG{o}{=}\PYG{p}{[}\PYG{l+s+s2}{\PYGZdq{}}\PYG{l+s+s2}{OLED}\PYG{l+s+s2}{\PYGZdq{}}\PYG{p}{,} \PYG{l+s+s2}{\PYGZdq{}}\PYG{l+s+s2}{Red}\PYG{l+s+s2}{\PYGZdq{}}\PYG{p}{]}\PYG{p}{,}
    \PYG{n}{length}\PYG{o}{=}\PYG{l+m+mf}{1e\PYGZhy{}3}\PYG{p}{,}
    \PYG{n}{width}\PYG{o}{=}\PYG{l+m+mf}{0.5e\PYGZhy{}3}
\PYG{p}{)}
\end{sphinxVerbatim}

\sphinxAtStartPar
Ein kreisförmiges Device (Radius 80µm) mit einem kreisförmigen
Ausschnitt (Radius 10µm) erstellen:

\begin{sphinxVerbatim}[commandchars=\\\{\}]
\PYG{n}{dev2} \PYG{o}{=} \PYG{n}{Device}\PYG{p}{(}
    \PYG{n}{name}\PYG{o}{=}\PYG{l+s+s2}{\PYGZdq{}}\PYG{l+s+s2}{Ring\PYGZus{}Struktur}\PYG{l+s+s2}{\PYGZdq{}}\PYG{p}{,}
    \PYG{n}{geometry}\PYG{o}{=}\PYG{l+s+s2}{\PYGZdq{}}\PYG{l+s+s2}{circle}\PYG{l+s+s2}{\PYGZdq{}}\PYG{p}{,}
    \PYG{n}{radius}\PYG{o}{=}\PYG{l+m+mf}{80e\PYGZhy{}6}\PYG{p}{,}
    \PYG{n}{cutout\PYGZus{}radius}\PYG{o}{=}\PYG{l+m+mf}{10e\PYGZhy{}6}
\PYG{p}{)}
\end{sphinxVerbatim}
\end{sphinxadmonition}

\end{fulllineitems}

\index{get\_area() (modules.device.DeviceManager.Device method)@\spxentry{get\_area()}\spxextra{modules.device.DeviceManager.Device method}}

\begin{fulllineitems}
\phantomsection\label{\detokenize{modulab/device_manager:modules.device.DeviceManager.Device.get_area}}
\pysigstartsignatures
\pysiglinewithargsret
{\sphinxbfcode{\sphinxupquote{get\_area}}}
{}
{}
\pysigstopsignatures
\sphinxAtStartPar
Berechnet die aktive Fläche {[}m\(\sp{\text{2}}\){]} des Devices.

\sphinxAtStartPar
Die Fläche wird als \sphinxtitleref{Hauptfläche \sphinxhyphen{} Ausschnittfläche} berechnet,
basierend auf der Geometrie und den \sphinxtitleref{dimensions}.
\begin{description}
\sphinxlineitem{Hinweis:}\begin{itemize}
\item {} 
\sphinxAtStartPar
Wenn die Maße ungültig (nicht numerisch) sind, wird 0.0 zurückgegeben.

\item {} 
\sphinxAtStartPar
Wenn die Ausschnittfläche größer als die Hauptfläche ist,
wird 0.0 zurückgegeben.

\end{itemize}

\end{description}
\begin{quote}\begin{description}
\sphinxlineitem{Returns}
\sphinxAtStartPar
Die berechnete aktive Fläche in Quadratmetern {[}m\(\sp{\text{2}}\){]}.

\sphinxlineitem{Return type}
\sphinxAtStartPar
float

\end{description}\end{quote}

\end{fulllineitems}

\index{to\_dict() (modules.device.DeviceManager.Device method)@\spxentry{to\_dict()}\spxextra{modules.device.DeviceManager.Device method}}

\begin{fulllineitems}
\phantomsection\label{\detokenize{modulab/device_manager:modules.device.DeviceManager.Device.to_dict}}
\pysigstartsignatures
\pysiglinewithargsret
{\sphinxbfcode{\sphinxupquote{to\_dict}}}
{}
{}
\pysigstopsignatures
\sphinxAtStartPar
Konvertiert das Device\sphinxhyphen{}Objekt in ein serialisierbares Diktat.

\sphinxAtStartPar
Wird verwendet, um das Device im ProfileManager zu speichern.
\begin{quote}\begin{description}
\sphinxlineitem{Returns}
\sphinxAtStartPar
Eine Diktat\sphinxhyphen{}Repräsentation des Devices.

\sphinxlineitem{Return type}
\sphinxAtStartPar
dict

\end{description}\end{quote}

\end{fulllineitems}

\index{from\_dict() (modules.device.DeviceManager.Device class method)@\spxentry{from\_dict()}\spxextra{modules.device.DeviceManager.Device class method}}

\begin{fulllineitems}
\phantomsection\label{\detokenize{modulab/device_manager:modules.device.DeviceManager.Device.from_dict}}
\pysigstartsignatures
\pysiglinewithargsret
{\sphinxbfcode{\sphinxupquote{\DUrole{k}{classmethod}\DUrole{w}{ }}}\sphinxbfcode{\sphinxupquote{from\_dict}}}
{\sphinxparam{\DUrole{n}{data}}}
{}
\pysigstopsignatures
\sphinxAtStartPar
Erstellt eine Device\sphinxhyphen{}Instanz aus einem Diktat.

\sphinxAtStartPar
Wird verwendet, um das Device aus dem ProfileManager zu laden.
\begin{quote}\begin{description}
\sphinxlineitem{Parameters}
\sphinxAtStartPar
\sphinxstyleliteralstrong{\sphinxupquote{data}} (\sphinxcode{\sphinxupquote{dict}}) \textendash{} Das Diktat, das \sphinxtitleref{to\_dict()} erzeugt hat.

\sphinxlineitem{Returns}
\sphinxAtStartPar
Eine neue Instanz der Device\sphinxhyphen{}Klasse.

\sphinxlineitem{Return type}
\sphinxAtStartPar
{\hyperref[\detokenize{modulab/device_manager:modules.device.DeviceManager.Device}]{\sphinxcrossref{Device}}}

\end{description}\end{quote}

\end{fulllineitems}

\index{\_\_repr\_\_() (modules.device.DeviceManager.Device method)@\spxentry{\_\_repr\_\_()}\spxextra{modules.device.DeviceManager.Device method}}

\begin{fulllineitems}
\phantomsection\label{\detokenize{modulab/device_manager:modules.device.DeviceManager.Device.__repr__}}
\pysigstartsignatures
\pysiglinewithargsret
{\sphinxbfcode{\sphinxupquote{\_\_repr\_\_}}}
{}
{}
\pysigstopsignatures
\sphinxAtStartPar
Stellt eine formale String\sphinxhyphen{}Repräsentation des Objekts bereit.

\end{fulllineitems}


\end{fulllineitems}

\index{DeviceManager (class in modules.device.DeviceManager)@\spxentry{DeviceManager}\spxextra{class in modules.device.DeviceManager}}

\begin{fulllineitems}
\phantomsection\label{\detokenize{modulab/device_manager:modules.device.DeviceManager.DeviceManager}}
\pysigstartsignatures
\pysiglinewithargsret
{\sphinxbfcode{\sphinxupquote{\DUrole{k}{class}\DUrole{w}{ }}}\sphinxcode{\sphinxupquote{modules.device.DeviceManager.}}\sphinxbfcode{\sphinxupquote{DeviceManager}}}
{\sphinxparam{\DUrole{n}{log\_manager}\DUrole{o}{=}\DUrole{default_value}{None}}\sphinxparamcomma \sphinxparam{\DUrole{n}{profile\_manager}\DUrole{o}{=}\DUrole{default_value}{None}}\sphinxparamcomma \sphinxparam{\DUrole{n}{parent}\DUrole{o}{=}\DUrole{default_value}{None}}}
{}
\pysigstopsignatures
\sphinxAtStartPar
Bases: \sphinxcode{\sphinxupquote{QObject}}

\sphinxAtStartPar
Erstellt, bearbeitet, löscht und verwaltet \sphinxtitleref{Device}\sphinxhyphen{}Objekte.

\sphinxAtStartPar
Diese Klasse dient als zentraler Service für die Verwaltung einer
Liste von Devices. Sie ist verantwortlich für:
\sphinxhyphen{} Das Erstellen, Bearbeiten und Löschen von \sphinxtitleref{Device}\sphinxhyphen{}Objekten.
\sphinxhyphen{} Das Speichern und Laden der Device\sphinxhyphen{}Liste im/aus dem \sphinxtitleref{ProfileManager}.
\sphinxhyphen{} Die Verwaltung, welches Device das “aktive” ist.
\begin{quote}\begin{description}
\sphinxlineitem{Parameters}\begin{itemize}
\item {} 
\sphinxAtStartPar
\sphinxstyleliteralstrong{\sphinxupquote{log\_manager}} (\sphinxcode{\sphinxupquote{LogManager}}) \textendash{} Eine Instanz des LogManagers.

\item {} 
\sphinxAtStartPar
\sphinxstyleliteralstrong{\sphinxupquote{profile\_manager}} (\sphinxcode{\sphinxupquote{ProfileManager}}) \textendash{} Eine Instanz des ProfileManagers.

\item {} 
\sphinxAtStartPar
\sphinxstyleliteralstrong{\sphinxupquote{parent}} (\sphinxcode{\sphinxupquote{QObject}}, \sphinxstyleemphasis{optional}) \textendash{} Ein übergeordnetes QObject für das Speichermanagement.

\end{itemize}

\end{description}\end{quote}
\begin{description}
\sphinxlineitem{Signale:}\begin{description}
\sphinxlineitem{device\_loaded (str):}
\sphinxAtStartPar
Wird ausgelöst, wenn ein Device mit \sphinxtitleref{set\_active\_device}
als aktiv ausgewählt wurde.
Args: (str: Der Name des geladenen Devices).

\end{description}

\end{description}
\index{KEY\_DEVICE\_LIST (modules.device.DeviceManager.DeviceManager attribute)@\spxentry{KEY\_DEVICE\_LIST}\spxextra{modules.device.DeviceManager.DeviceManager attribute}}

\begin{fulllineitems}
\phantomsection\label{\detokenize{modulab/device_manager:modules.device.DeviceManager.DeviceManager.KEY_DEVICE_LIST}}
\pysigstartsignatures
\pysigline
{\sphinxbfcode{\sphinxupquote{KEY\_DEVICE\_LIST}}\sphinxbfcode{\sphinxupquote{\DUrole{w}{ }\DUrole{p}{=}\DUrole{w}{ }\textquotesingle{}devices\textquotesingle{}}}}
\pysigstopsignatures
\sphinxAtStartPar
Der Profil\sphinxhyphen{}Schlüssel zum Speichern der Liste aller Device\sphinxhyphen{}Diktate.
\begin{quote}\begin{description}
\sphinxlineitem{Type}
\sphinxAtStartPar
str

\end{description}\end{quote}

\end{fulllineitems}

\index{KEY\_ACTIVE\_DEVICE (modules.device.DeviceManager.DeviceManager attribute)@\spxentry{KEY\_ACTIVE\_DEVICE}\spxextra{modules.device.DeviceManager.DeviceManager attribute}}

\begin{fulllineitems}
\phantomsection\label{\detokenize{modulab/device_manager:modules.device.DeviceManager.DeviceManager.KEY_ACTIVE_DEVICE}}
\pysigstartsignatures
\pysigline
{\sphinxbfcode{\sphinxupquote{KEY\_ACTIVE\_DEVICE}}\sphinxbfcode{\sphinxupquote{\DUrole{w}{ }\DUrole{p}{=}\DUrole{w}{ }\textquotesingle{}active\_device\_name\textquotesingle{}}}}
\pysigstopsignatures
\sphinxAtStartPar
Der Profil\sphinxhyphen{}Schlüssel zum Speichern des Namens des aktiven Devices.
\begin{quote}\begin{description}
\sphinxlineitem{Type}
\sphinxAtStartPar
str

\end{description}\end{quote}

\end{fulllineitems}

\index{\_\_init\_\_() (modules.device.DeviceManager.DeviceManager method)@\spxentry{\_\_init\_\_()}\spxextra{modules.device.DeviceManager.DeviceManager method}}

\begin{fulllineitems}
\phantomsection\label{\detokenize{modulab/device_manager:modules.device.DeviceManager.DeviceManager.__init__}}
\pysigstartsignatures
\pysiglinewithargsret
{\sphinxbfcode{\sphinxupquote{\_\_init\_\_}}}
{\sphinxparam{\DUrole{n}{log\_manager}\DUrole{o}{=}\DUrole{default_value}{None}}\sphinxparamcomma \sphinxparam{\DUrole{n}{profile\_manager}\DUrole{o}{=}\DUrole{default_value}{None}}\sphinxparamcomma \sphinxparam{\DUrole{n}{parent}\DUrole{o}{=}\DUrole{default_value}{None}}}
{}
\pysigstopsignatures
\sphinxAtStartPar
Initialisiert den DeviceManager.
\begin{quote}\begin{description}
\sphinxlineitem{Parameters}\begin{itemize}
\item {} 
\sphinxAtStartPar
\sphinxstyleliteralstrong{\sphinxupquote{log\_manager}} (\sphinxcode{\sphinxupquote{LogManager}}, \sphinxstyleemphasis{optional}) \textendash{} Die LogManager\sphinxhyphen{}Instanz.

\item {} 
\sphinxAtStartPar
\sphinxstyleliteralstrong{\sphinxupquote{profile\_manager}} (\sphinxcode{\sphinxupquote{ProfileManager}}, \sphinxstyleemphasis{optional}) \textendash{} Die ProfileManager\sphinxhyphen{}Instanz.

\item {} 
\sphinxAtStartPar
\sphinxstyleliteralstrong{\sphinxupquote{parent}} (\sphinxcode{\sphinxupquote{QObject}}, \sphinxstyleemphasis{optional}) \textendash{} Ein übergeordnetes QObject.

\end{itemize}

\end{description}\end{quote}

\end{fulllineitems}

\index{load\_from\_profile() (modules.device.DeviceManager.DeviceManager method)@\spxentry{load\_from\_profile()}\spxextra{modules.device.DeviceManager.DeviceManager method}}

\begin{fulllineitems}
\phantomsection\label{\detokenize{modulab/device_manager:modules.device.DeviceManager.DeviceManager.load_from_profile}}
\pysigstartsignatures
\pysiglinewithargsret
{\sphinxbfcode{\sphinxupquote{load\_from\_profile}}}
{}
{}
\pysigstopsignatures
\sphinxAtStartPar
Lädt die Device\sphinxhyphen{}Liste und das aktive Device aus dem geladenen Profil.

\sphinxAtStartPar
Diese Funktion wird typischerweise aufgerufen, nachdem der
ProfileManager das \sphinxtitleref{profile\_loaded}\sphinxhyphen{}Signal gesendet hat.

\end{fulllineitems}

\index{create\_device() (modules.device.DeviceManager.DeviceManager method)@\spxentry{create\_device()}\spxextra{modules.device.DeviceManager.DeviceManager method}}

\begin{fulllineitems}
\phantomsection\label{\detokenize{modulab/device_manager:modules.device.DeviceManager.DeviceManager.create_device}}
\pysigstartsignatures
\pysiglinewithargsret
{\sphinxbfcode{\sphinxupquote{create\_device}}}
{\sphinxparam{\DUrole{n}{name}}\sphinxparamcomma \sphinxparam{\DUrole{n}{geometry}}\sphinxparamcomma \sphinxparam{\DUrole{n}{tags}\DUrole{o}{=}\DUrole{default_value}{None}}\sphinxparamcomma \sphinxparam{\DUrole{o}{**}\DUrole{n}{dimensions}}}
{}
\pysigstopsignatures
\sphinxAtStartPar
Erstellt ein neues Device, fügt es zur Liste hinzu und speichert es im Profil.
\begin{quote}\begin{description}
\sphinxlineitem{Parameters}\begin{itemize}
\item {} 
\sphinxAtStartPar
\sphinxstyleliteralstrong{\sphinxupquote{name}} (\sphinxcode{\sphinxupquote{str}}) \textendash{} Der eindeutige Name des Devices (z.B. “Pixel\_R1C1”).

\item {} 
\sphinxAtStartPar
\sphinxstyleliteralstrong{\sphinxupquote{geometry}} (\sphinxcode{\sphinxupquote{str}}) \textendash{} Der Geometrie\sphinxhyphen{}Typ (‘rectangle’ oder ‘circle’).

\item {} 
\sphinxAtStartPar
\sphinxstyleliteralstrong{\sphinxupquote{tags}} (\sphinxcode{\sphinxupquote{list}}, \sphinxstyleemphasis{optional}) \textendash{} Eine Liste von String\sphinxhyphen{}Tags.

\item {} 
\sphinxAtStartPar
\sphinxstyleliteralstrong{\sphinxupquote{**dimensions}} (\sphinxcode{\sphinxupquote{float}}) \textendash{} Dynamische Schlüsselwortargumente für die Maße
des Devices in Metern {[}m{]} (z.B. \sphinxtitleref{length=1e\sphinxhyphen{}3}).

\end{itemize}

\sphinxlineitem{Returns}
\sphinxAtStartPar
True bei Erfolg, False, wenn der Name bereits existiert.

\sphinxlineitem{Return type}
\sphinxAtStartPar
bool

\end{description}\end{quote}

\begin{sphinxadmonition}{note}{Examples}

\sphinxAtStartPar
Ein rechteckiges Device (1mm x 0.5mm) erstellen:

\begin{sphinxVerbatim}[commandchars=\\\{\}]
\PYG{c+c1}{\PYGZsh{} Annahme: \PYGZsq{}device\PYGZus{}mgr\PYGZsq{} ist eine Instanz von DeviceManager}
\PYG{n}{device\PYGZus{}mgr}\PYG{o}{.}\PYG{n}{create\PYGZus{}device}\PYG{p}{(}
    \PYG{l+s+s2}{\PYGZdq{}}\PYG{l+s+s2}{OLED\PYGZus{}Pixel\PYGZus{}1}\PYG{l+s+s2}{\PYGZdq{}}\PYG{p}{,}
    \PYG{l+s+s2}{\PYGZdq{}}\PYG{l+s+s2}{rectangle}\PYG{l+s+s2}{\PYGZdq{}}\PYG{p}{,}
    \PYG{n}{length}\PYG{o}{=}\PYG{l+m+mf}{1e\PYGZhy{}3}\PYG{p}{,}
    \PYG{n}{width}\PYG{o}{=}\PYG{l+m+mf}{0.5e\PYGZhy{}3}
\PYG{p}{)}
\end{sphinxVerbatim}

\sphinxAtStartPar
Ein kreisförmiges Device (Radius 80µm) mit Tags erstellen:

\begin{sphinxVerbatim}[commandchars=\\\{\}]
\PYG{n}{device\PYGZus{}mgr}\PYG{o}{.}\PYG{n}{create\PYGZus{}device}\PYG{p}{(}
    \PYG{l+s+s2}{\PYGZdq{}}\PYG{l+s+s2}{Pin80}\PYG{l+s+s2}{\PYGZdq{}}\PYG{p}{,}
    \PYG{l+s+s2}{\PYGZdq{}}\PYG{l+s+s2}{circle}\PYG{l+s+s2}{\PYGZdq{}}\PYG{p}{,}
    \PYG{n}{tags}\PYG{o}{=}\PYG{p}{[}\PYG{l+s+s2}{\PYGZdq{}}\PYG{l+s+s2}{Test\PYGZus{}Struktur}\PYG{l+s+s2}{\PYGZdq{}}\PYG{p}{,} \PYG{l+s+s2}{\PYGZdq{}}\PYG{l+s+s2}{Rund}\PYG{l+s+s2}{\PYGZdq{}}\PYG{p}{]}\PYG{p}{,}
    \PYG{n}{radius}\PYG{o}{=}\PYG{l+m+mf}{80e\PYGZhy{}6}
\PYG{p}{)}
\end{sphinxVerbatim}
\end{sphinxadmonition}

\end{fulllineitems}

\index{delete\_device() (modules.device.DeviceManager.DeviceManager method)@\spxentry{delete\_device()}\spxextra{modules.device.DeviceManager.DeviceManager method}}

\begin{fulllineitems}
\phantomsection\label{\detokenize{modulab/device_manager:modules.device.DeviceManager.DeviceManager.delete_device}}
\pysigstartsignatures
\pysiglinewithargsret
{\sphinxbfcode{\sphinxupquote{delete\_device}}}
{\sphinxparam{\DUrole{n}{name}}}
{}
\pysigstopsignatures
\sphinxAtStartPar
Löscht ein Device anhand seines Namens aus der Liste.
\begin{quote}\begin{description}
\sphinxlineitem{Parameters}
\sphinxAtStartPar
\sphinxstyleliteralstrong{\sphinxupquote{name}} (\sphinxcode{\sphinxupquote{str}}) \textendash{} Der Name des zu löschenden Devices.

\sphinxlineitem{Returns}
\sphinxAtStartPar
True bei Erfolg, False, wenn das Device nicht gefunden wurde.

\sphinxlineitem{Return type}
\sphinxAtStartPar
bool

\end{description}\end{quote}

\end{fulllineitems}

\index{edit\_device() (modules.device.DeviceManager.DeviceManager method)@\spxentry{edit\_device()}\spxextra{modules.device.DeviceManager.DeviceManager method}}

\begin{fulllineitems}
\phantomsection\label{\detokenize{modulab/device_manager:modules.device.DeviceManager.DeviceManager.edit_device}}
\pysigstartsignatures
\pysiglinewithargsret
{\sphinxbfcode{\sphinxupquote{edit\_device}}}
{\sphinxparam{\DUrole{n}{name}}\sphinxparamcomma \sphinxparam{\DUrole{n}{new\_geometry}\DUrole{o}{=}\DUrole{default_value}{None}}\sphinxparamcomma \sphinxparam{\DUrole{n}{new\_tags}\DUrole{o}{=}\DUrole{default_value}{None}}\sphinxparamcomma \sphinxparam{\DUrole{n}{new\_dimensions}\DUrole{o}{=}\DUrole{default_value}{None}}}
{}
\pysigstopsignatures
\sphinxAtStartPar
Bearbeitet ein existierendes Device.

\sphinxAtStartPar
Der Name kann nicht geändert werden. Nur die übergebenen (nicht\sphinxhyphen{}None)
Parameter werden aktualisiert.
\begin{quote}\begin{description}
\sphinxlineitem{Parameters}\begin{itemize}
\item {} 
\sphinxAtStartPar
\sphinxstyleliteralstrong{\sphinxupquote{name}} (\sphinxcode{\sphinxupquote{str}}) \textendash{} Der Name des zu bearbeitenden Devices.

\item {} 
\sphinxAtStartPar
\sphinxstyleliteralstrong{\sphinxupquote{new\_geometry}} (\sphinxcode{\sphinxupquote{str}}, \sphinxstyleemphasis{optional}) \textendash{} Die neue Geometrie (‘rectangle’/’circle’).

\item {} 
\sphinxAtStartPar
\sphinxstyleliteralstrong{\sphinxupquote{new\_tags}} (\sphinxcode{\sphinxupquote{list}}, \sphinxstyleemphasis{optional}) \textendash{} Die \sphinxstyleemphasis{komplett neue} Liste von Tags.

\item {} 
\sphinxAtStartPar
\sphinxstyleliteralstrong{\sphinxupquote{new\_dimensions}} (\sphinxcode{\sphinxupquote{dict}}, \sphinxstyleemphasis{optional}) \textendash{} Das \sphinxstyleemphasis{komplett neue} Diktat
für Dimensionen.

\end{itemize}

\sphinxlineitem{Returns}
\sphinxAtStartPar
True bei Erfolg, False, wenn das Device nicht gefunden wurde.

\sphinxlineitem{Return type}
\sphinxAtStartPar
bool

\end{description}\end{quote}

\begin{sphinxadmonition}{note}{Examples}

\sphinxAtStartPar
Die Dimensionen eines Devices ändern:

\begin{sphinxVerbatim}[commandchars=\\\{\}]
\PYG{n}{neue\PYGZus{}maße} \PYG{o}{=} \PYG{p}{\PYGZob{}}\PYG{l+s+s1}{\PYGZsq{}}\PYG{l+s+s1}{length}\PYG{l+s+s1}{\PYGZsq{}}\PYG{p}{:} \PYG{l+m+mf}{1.1e\PYGZhy{}3}\PYG{p}{,} \PYG{l+s+s1}{\PYGZsq{}}\PYG{l+s+s1}{width}\PYG{l+s+s1}{\PYGZsq{}}\PYG{p}{:} \PYG{l+m+mf}{0.6e\PYGZhy{}3}\PYG{p}{\PYGZcb{}}
\PYG{n}{device\PYGZus{}mgr}\PYG{o}{.}\PYG{n}{edit\PYGZus{}device}\PYG{p}{(}\PYG{l+s+s2}{\PYGZdq{}}\PYG{l+s+s2}{OLED\PYGZus{}Pixel\PYGZus{}1}\PYG{l+s+s2}{\PYGZdq{}}\PYG{p}{,} \PYG{n}{new\PYGZus{}dimensions}\PYG{o}{=}\PYG{n}{neue\PYGZus{}maße}\PYG{p}{)}
\end{sphinxVerbatim}

\sphinxAtStartPar
Nur die Tags eines Devices ändern:

\begin{sphinxVerbatim}[commandchars=\\\{\}]
\PYG{n}{device\PYGZus{}mgr}\PYG{o}{.}\PYG{n}{edit\PYGZus{}device}\PYG{p}{(}\PYG{l+s+s2}{\PYGZdq{}}\PYG{l+s+s2}{Pin80}\PYG{l+s+s2}{\PYGZdq{}}\PYG{p}{,} \PYG{n}{new\PYGZus{}tags}\PYG{o}{=}\PYG{p}{[}\PYG{l+s+s2}{\PYGZdq{}}\PYG{l+s+s2}{Test\PYGZus{}Struktur}\PYG{l+s+s2}{\PYGZdq{}}\PYG{p}{,} \PYG{l+s+s2}{\PYGZdq{}}\PYG{l+s+s2}{Wichtig}\PYG{l+s+s2}{\PYGZdq{}}\PYG{p}{]}\PYG{p}{)}
\end{sphinxVerbatim}
\end{sphinxadmonition}

\end{fulllineitems}

\index{get\_device\_by\_name() (modules.device.DeviceManager.DeviceManager method)@\spxentry{get\_device\_by\_name()}\spxextra{modules.device.DeviceManager.DeviceManager method}}

\begin{fulllineitems}
\phantomsection\label{\detokenize{modulab/device_manager:modules.device.DeviceManager.DeviceManager.get_device_by_name}}
\pysigstartsignatures
\pysiglinewithargsret
{\sphinxbfcode{\sphinxupquote{get\_device\_by\_name}}}
{\sphinxparam{\DUrole{n}{name}}}
{}
\pysigstopsignatures
\sphinxAtStartPar
Sucht und retourniert das Device\sphinxhyphen{}Objekt anhand des Namens.
\begin{quote}\begin{description}
\sphinxlineitem{Parameters}
\sphinxAtStartPar
\sphinxstyleliteralstrong{\sphinxupquote{name}} (\sphinxcode{\sphinxupquote{str}}) \textendash{} Der Name des gesuchten Devices.

\sphinxlineitem{Returns}
\sphinxAtStartPar
Das \sphinxtitleref{Device}\sphinxhyphen{}Objekt oder \sphinxtitleref{None}, wenn nicht gefunden.

\sphinxlineitem{Return type}
\sphinxAtStartPar
{\hyperref[\detokenize{modulab/device_manager:modules.device.DeviceManager.Device}]{\sphinxcrossref{Device}}} | None

\end{description}\end{quote}

\end{fulllineitems}

\index{list\_device\_names() (modules.device.DeviceManager.DeviceManager method)@\spxentry{list\_device\_names()}\spxextra{modules.device.DeviceManager.DeviceManager method}}

\begin{fulllineitems}
\phantomsection\label{\detokenize{modulab/device_manager:modules.device.DeviceManager.DeviceManager.list_device_names}}
\pysigstartsignatures
\pysiglinewithargsret
{\sphinxbfcode{\sphinxupquote{list\_device\_names}}}
{}
{}
\pysigstopsignatures
\sphinxAtStartPar
Gibt eine Liste aller Device\sphinxhyphen{}Namen zurück.

\sphinxAtStartPar
Nützlich für die Anzeige in einer Combobox oder Liste in der GUI.
\begin{quote}\begin{description}
\sphinxlineitem{Returns}
\sphinxAtStartPar
Eine Liste der Namen aller geladenen Devices.

\sphinxlineitem{Return type}
\sphinxAtStartPar
list{[}str{]}

\end{description}\end{quote}

\begin{sphinxadmonition}{note}{Examples}

\sphinxAtStartPar
Eine QComboBox mit allen Device\sphinxhyphen{}Namen füllen:

\begin{sphinxVerbatim}[commandchars=\\\{\}]
\PYG{n}{namen} \PYG{o}{=} \PYG{n}{device\PYGZus{}mgr}\PYG{o}{.}\PYG{n}{list\PYGZus{}device\PYGZus{}names}\PYG{p}{(}\PYG{p}{)}
\PYG{n}{ui}\PYG{o}{.}\PYG{n}{device\PYGZus{}combobox}\PYG{o}{.}\PYG{n}{clear}\PYG{p}{(}\PYG{p}{)}
\PYG{n}{ui}\PYG{o}{.}\PYG{n}{device\PYGZus{}combobox}\PYG{o}{.}\PYG{n}{addItems}\PYG{p}{(}\PYG{n}{namen}\PYG{p}{)}
\end{sphinxVerbatim}
\end{sphinxadmonition}

\end{fulllineitems}

\index{set\_active\_device() (modules.device.DeviceManager.DeviceManager method)@\spxentry{set\_active\_device()}\spxextra{modules.device.DeviceManager.DeviceManager method}}

\begin{fulllineitems}
\phantomsection\label{\detokenize{modulab/device_manager:modules.device.DeviceManager.DeviceManager.set_active_device}}
\pysigstartsignatures
\pysiglinewithargsret
{\sphinxbfcode{\sphinxupquote{set\_active\_device}}}
{\sphinxparam{\DUrole{n}{name}}}
{}
\pysigstopsignatures
\sphinxAtStartPar
Setzt das aktive Device für die Anwendung.

\sphinxAtStartPar
Löst das \sphinxtitleref{device\_loaded}\sphinxhyphen{}Signal aus.
\begin{quote}\begin{description}
\sphinxlineitem{Parameters}
\sphinxAtStartPar
\sphinxstyleliteralstrong{\sphinxupquote{name}} (\sphinxcode{\sphinxupquote{str}}) \textendash{} Der Name des Devices, das aktiv werden soll.

\sphinxlineitem{Returns}
\sphinxAtStartPar
True bei Erfolg, False, wenn das Device nicht gefunden wurde.

\sphinxlineitem{Return type}
\sphinxAtStartPar
bool

\end{description}\end{quote}

\end{fulllineitems}

\index{get\_active\_device() (modules.device.DeviceManager.DeviceManager method)@\spxentry{get\_active\_device()}\spxextra{modules.device.DeviceManager.DeviceManager method}}

\begin{fulllineitems}
\phantomsection\label{\detokenize{modulab/device_manager:modules.device.DeviceManager.DeviceManager.get_active_device}}
\pysigstartsignatures
\pysiglinewithargsret
{\sphinxbfcode{\sphinxupquote{get\_active\_device}}}
{}
{}
\pysigstopsignatures
\sphinxAtStartPar
Gibt das \sphinxstyleemphasis{gesamte Objekt} des aktiven Devices zurück.

\sphinxAtStartPar
Die UI kann dann \sphinxtitleref{device.get\_area()} oder \sphinxtitleref{device.dimensions}
selbst aufrufen.
\begin{quote}\begin{description}
\sphinxlineitem{Returns}
\sphinxAtStartPar
Das aktuell aktive \sphinxtitleref{Device}\sphinxhyphen{}Objekt oder \sphinxtitleref{None}.

\sphinxlineitem{Return type}
\sphinxAtStartPar
{\hyperref[\detokenize{modulab/device_manager:modules.device.DeviceManager.Device}]{\sphinxcrossref{Device}}} | None

\end{description}\end{quote}

\begin{sphinxadmonition}{note}{Examples}

\sphinxAtStartPar
Informationen des aktiven Devices abrufen:

\begin{sphinxVerbatim}[commandchars=\\\{\}]
\PYG{n}{aktives\PYGZus{}gerät} \PYG{o}{=} \PYG{n}{device\PYGZus{}mgr}\PYG{o}{.}\PYG{n}{get\PYGZus{}active\PYGZus{}device}\PYG{p}{(}\PYG{p}{)}
\PYG{k}{if} \PYG{n}{aktives\PYGZus{}gerät}\PYG{p}{:}
    \PYG{n+nb}{print}\PYG{p}{(}\PYG{l+s+sa}{f}\PYG{l+s+s2}{\PYGZdq{}}\PYG{l+s+s2}{Aktuell: }\PYG{l+s+si}{\PYGZob{}}\PYG{n}{aktives\PYGZus{}gerät}\PYG{o}{.}\PYG{n}{name}\PYG{l+s+si}{\PYGZcb{}}\PYG{l+s+s2}{\PYGZdq{}}\PYG{p}{)}
    \PYG{n+nb}{print}\PYG{p}{(}\PYG{l+s+sa}{f}\PYG{l+s+s2}{\PYGZdq{}}\PYG{l+s+s2}{Fläche: }\PYG{l+s+si}{\PYGZob{}}\PYG{n}{aktives\PYGZus{}gerät}\PYG{o}{.}\PYG{n}{get\PYGZus{}area}\PYG{p}{(}\PYG{p}{)}\PYG{l+s+si}{\PYGZcb{}}\PYG{l+s+s2}{ m\(\sp{\text{2}}\)}\PYG{l+s+s2}{\PYGZdq{}}\PYG{p}{)}
\PYG{k}{else}\PYG{p}{:}
    \PYG{n+nb}{print}\PYG{p}{(}\PYG{l+s+s2}{\PYGZdq{}}\PYG{l+s+s2}{Kein Device aktiv.}\PYG{l+s+s2}{\PYGZdq{}}\PYG{p}{)}
\end{sphinxVerbatim}
\end{sphinxadmonition}

\end{fulllineitems}

\index{get\_active\_device\_area() (modules.device.DeviceManager.DeviceManager method)@\spxentry{get\_active\_device\_area()}\spxextra{modules.device.DeviceManager.DeviceManager method}}

\begin{fulllineitems}
\phantomsection\label{\detokenize{modulab/device_manager:modules.device.DeviceManager.DeviceManager.get_active_device_area}}
\pysigstartsignatures
\pysiglinewithargsret
{\sphinxbfcode{\sphinxupquote{get\_active\_device\_area}}}
{}
{}
\pysigstopsignatures
\sphinxAtStartPar
Bequemlichkeitsfunktion: Gibt die Fläche {[}m\(\sp{\text{2}}\){]} des aktiven Devices zurück.
\begin{quote}\begin{description}
\sphinxlineitem{Returns}
\sphinxAtStartPar
Die Fläche in m\(\sp{\text{2}}\) oder \sphinxtitleref{None}, wenn kein Device aktiv ist.

\sphinxlineitem{Return type}
\sphinxAtStartPar
float | None

\end{description}\end{quote}

\end{fulllineitems}

\index{get\_active\_device\_dimensions() (modules.device.DeviceManager.DeviceManager method)@\spxentry{get\_active\_device\_dimensions()}\spxextra{modules.device.DeviceManager.DeviceManager method}}

\begin{fulllineitems}
\phantomsection\label{\detokenize{modulab/device_manager:modules.device.DeviceManager.DeviceManager.get_active_device_dimensions}}
\pysigstartsignatures
\pysiglinewithargsret
{\sphinxbfcode{\sphinxupquote{get\_active\_device\_dimensions}}}
{}
{}
\pysigstopsignatures
\sphinxAtStartPar
Bequemlichkeitsfunktion: Gibt die Maße des aktiven Devices zurück.
\begin{quote}\begin{description}
\sphinxlineitem{Returns}
\sphinxAtStartPar
\begin{description}
\sphinxlineitem{Das Dimensions\sphinxhyphen{}Wörterbuch (z.B. \sphinxtitleref{\{‘length’: 0.1, …\}})}
\sphinxAtStartPar
oder ein leeres dict.

\end{description}


\sphinxlineitem{Return type}
\sphinxAtStartPar
dict

\end{description}\end{quote}

\end{fulllineitems}


\end{fulllineitems}


\sphinxstepscope


\chapter{SMU Manager}
\label{\detokenize{modulab/smu_manager:module-modules.smu.SmuManager}}\label{\detokenize{modulab/smu_manager:id1}}\label{\detokenize{modulab/smu_manager:smu-manager}}\label{\detokenize{modulab/smu_manager::doc}}\index{module@\spxentry{module}!modules.smu.SmuManager@\spxentry{modules.smu.SmuManager}}\index{modules.smu.SmuManager@\spxentry{modules.smu.SmuManager}!module@\spxentry{module}}\index{SmuManager (class in modules.smu.SmuManager)@\spxentry{SmuManager}\spxextra{class in modules.smu.SmuManager}}

\begin{fulllineitems}
\phantomsection\label{\detokenize{modulab/smu_manager:modules.smu.SmuManager.SmuManager}}
\pysigstartsignatures
\pysiglinewithargsret
{\sphinxbfcode{\sphinxupquote{\DUrole{k}{class}\DUrole{w}{ }}}\sphinxcode{\sphinxupquote{modules.smu.SmuManager.}}\sphinxbfcode{\sphinxupquote{SmuManager}}}
{\sphinxparam{\DUrole{n}{log\_manager}}\sphinxparamcomma \sphinxparam{\DUrole{n}{profile\_manager}}}
{}
\pysigstopsignatures
\sphinxAtStartPar
Bases: \sphinxcode{\sphinxupquote{QObject}}

\sphinxAtStartPar
Manager zur Steuerung und Verwaltung von SMU\sphinxhyphen{}Geräten (Source Measure Units).

\sphinxAtStartPar
Diese Klasse kapselt die Gerätetreiber (z.B. Keithley2602), verwaltet die
serielle Verbindung, aktualisiert die Geräteliste und stellt High\sphinxhyphen{}Level\sphinxhyphen{}Methoden
für die Konfiguration (Source, Sense, Limits) und Messung (IV) bereit.
Sie nutzt PySide6\sphinxhyphen{}Signale, um die GUI über Statusänderungen zu informieren.
\begin{quote}\begin{description}
\sphinxlineitem{Parameters}\begin{itemize}
\item {} 
\sphinxAtStartPar
\sphinxstyleliteralstrong{\sphinxupquote{log\_manager}} (\sphinxcode{\sphinxupquote{LogManager}}) \textendash{} Eine Instanz eines Log\sphinxhyphen{}Managers (erwartet .info, .error, etc.).

\item {} 
\sphinxAtStartPar
\sphinxstyleliteralstrong{\sphinxupquote{profile\_manager}} (\sphinxcode{\sphinxupquote{ProfileManager}}) \textendash{} Eine Instanz zur Verwaltung von
App\sphinxhyphen{}Einstellungen (Lesen/Schreiben).

\end{itemize}

\end{description}\end{quote}
\begin{description}
\sphinxlineitem{Signale:}\begin{description}
\sphinxlineitem{connection\_status\_changed (bool, str):}
\sphinxAtStartPar
Wird ausgelöst, wenn sich der Verbindungsstatus ändert.
Args: (bool: verbunden, str: Gerätename/IDN).

\sphinxlineitem{device\_list\_updated (list):}
\sphinxAtStartPar
Wird ausgelöst, nachdem die Liste der seriellen Ports aktualisiert wurde.
Args: (list: Liste von Port\sphinxhyphen{}Namen {[}str{]}).

\sphinxlineitem{new\_measurement\_acquired (str, float, float):}
\sphinxAtStartPar
Wird ausgelöst, wenn eine neue Messung verfügbar ist.
Args: (str: Kanal, float: Strom, float: Spannung).

\end{description}

\end{description}
\index{get\_deviceList() (modules.smu.SmuManager.SmuManager method)@\spxentry{get\_deviceList()}\spxextra{modules.smu.SmuManager.SmuManager method}}

\begin{fulllineitems}
\phantomsection\label{\detokenize{modulab/smu_manager:modules.smu.SmuManager.SmuManager.get_deviceList}}
\pysigstartsignatures
\pysiglinewithargsret
{\sphinxbfcode{\sphinxupquote{get\_deviceList}}}
{}
{}
\pysigstopsignatures
\sphinxAtStartPar
Scannt nach verfügbaren seriellen Ports und aktualisiert die interne Liste.

\sphinxAtStartPar
Sucht nach allen COM\sphinxhyphen{}Ports und fügt zusätzlich einen “DUMMY”\sphinxhyphen{}Port
für Testzwecke hinzu. Löst das \sphinxtitleref{device\_list\_updated}\sphinxhyphen{}Signal aus.
\begin{quote}\begin{description}
\sphinxlineitem{Returns}
\sphinxAtStartPar
Eine Liste der gefundenen Port\sphinxhyphen{}Namen (z.B. {[}‘COM1’, ‘COM3’, ‘DUMMY’{]}).

\sphinxlineitem{Return type}
\sphinxAtStartPar
list

\end{description}\end{quote}

\end{fulllineitems}

\index{connect() (modules.smu.SmuManager.SmuManager method)@\spxentry{connect()}\spxextra{modules.smu.SmuManager.SmuManager method}}

\begin{fulllineitems}
\phantomsection\label{\detokenize{modulab/smu_manager:modules.smu.SmuManager.SmuManager.connect}}
\pysigstartsignatures
\pysiglinewithargsret
{\sphinxbfcode{\sphinxupquote{connect}}}
{\sphinxparam{\DUrole{n}{port\_name}}}
{}
\pysigstopsignatures
\sphinxAtStartPar
Verbindet eine SMU an einem bestimmten COM\sphinxhyphen{}Port.
…
\begin{quote}\begin{description}
\sphinxlineitem{Parameters}
\sphinxAtStartPar
\sphinxstyleliteralstrong{\sphinxupquote{port\_name}} (\sphinxcode{\sphinxupquote{str}}) \textendash{} Der Name des Ports (z.B. “COM1” oder “DUMMY”).

\sphinxlineitem{Returns}
\sphinxAtStartPar
True bei erfolgreicher Verbindung, sonst False.

\sphinxlineitem{Return type}
\sphinxAtStartPar
bool

\end{description}\end{quote}

\begin{sphinxadmonition}{note}{Examples}

\sphinxAtStartPar
Mit einem echten COM\sphinxhyphen{}Port verbinden:

\begin{sphinxVerbatim}[commandchars=\\\{\}]
\PYG{c+c1}{\PYGZsh{} \PYGZsq{}COM3\PYGZsq{} ist nur ein Beispiel}
\PYG{n}{success} \PYG{o}{=} \PYG{n}{manager}\PYG{o}{.}\PYG{n}{connect}\PYG{p}{(}\PYG{l+s+s1}{\PYGZsq{}}\PYG{l+s+s1}{COM3}\PYG{l+s+s1}{\PYGZsq{}}\PYG{p}{)}
\PYG{k}{if} \PYG{n}{success}\PYG{p}{:}
    \PYG{n+nb}{print}\PYG{p}{(}\PYG{l+s+s2}{\PYGZdq{}}\PYG{l+s+s2}{Verbunden!}\PYG{l+s+s2}{\PYGZdq{}}\PYG{p}{)}
\end{sphinxVerbatim}

\sphinxAtStartPar
Mit dem DUMMY\sphinxhyphen{}Gerät für Tests verbinden:

\begin{sphinxVerbatim}[commandchars=\\\{\}]
\PYG{n}{manager}\PYG{o}{.}\PYG{n}{connect}\PYG{p}{(}\PYG{l+s+s1}{\PYGZsq{}}\PYG{l+s+s1}{DUMMY}\PYG{l+s+s1}{\PYGZsq{}}\PYG{p}{)}
\end{sphinxVerbatim}
\end{sphinxadmonition}

\end{fulllineitems}

\index{connect\_LastDevice() (modules.smu.SmuManager.SmuManager method)@\spxentry{connect\_LastDevice()}\spxextra{modules.smu.SmuManager.SmuManager method}}

\begin{fulllineitems}
\phantomsection\label{\detokenize{modulab/smu_manager:modules.smu.SmuManager.SmuManager.connect_LastDevice}}
\pysigstartsignatures
\pysiglinewithargsret
{\sphinxbfcode{\sphinxupquote{connect\_LastDevice}}}
{}
{}
\pysigstopsignatures
\sphinxAtStartPar
Versucht, die Verbindung mit dem zuletzt genutzten Gerät wiederherzustellen.

\sphinxAtStartPar
Aktualisiert zuerst die Geräteliste und prüft, ob der gespeicherte
Port (\sphinxtitleref{self.LastDevice}) verfügbar ist.
\begin{quote}\begin{description}
\sphinxlineitem{Returns}
\sphinxAtStartPar
\begin{description}
\sphinxlineitem{True bei Erfolg, False, wenn kein Gerät gespeichert war}
\sphinxAtStartPar
oder die Verbindung fehlschlägt.

\end{description}


\sphinxlineitem{Return type}
\sphinxAtStartPar
bool

\end{description}\end{quote}

\end{fulllineitems}

\index{disconnect() (modules.smu.SmuManager.SmuManager method)@\spxentry{disconnect()}\spxextra{modules.smu.SmuManager.SmuManager method}}

\begin{fulllineitems}
\phantomsection\label{\detokenize{modulab/smu_manager:modules.smu.SmuManager.SmuManager.disconnect}}
\pysigstartsignatures
\pysiglinewithargsret
{\sphinxbfcode{\sphinxupquote{disconnect}}}
{}
{}
\pysigstopsignatures
\sphinxAtStartPar
Trennt die aktive Verbindung zum SMU\sphinxhyphen{}Gerät.

\sphinxAtStartPar
Setzt den internen Zustand zurück und löst \sphinxtitleref{connection\_status\_changed} aus.

\end{fulllineitems}

\index{get\_activeDeviceName() (modules.smu.SmuManager.SmuManager method)@\spxentry{get\_activeDeviceName()}\spxextra{modules.smu.SmuManager.SmuManager method}}

\begin{fulllineitems}
\phantomsection\label{\detokenize{modulab/smu_manager:modules.smu.SmuManager.SmuManager.get_activeDeviceName}}
\pysigstartsignatures
\pysiglinewithargsret
{\sphinxbfcode{\sphinxupquote{get\_activeDeviceName}}}
{}
{}
\pysigstopsignatures
\sphinxAtStartPar
Gibt einen formatierten Namen des aktuell verbundenen Geräts zurück.

\sphinxAtStartPar
Parst die IDN\sphinxhyphen{}Nachricht, um Modell und Seriennummer zu extrahieren.
\begin{quote}\begin{description}
\sphinxlineitem{Returns}
\sphinxAtStartPar
\begin{description}
\sphinxlineitem{Der formatierte Gerätename}
\sphinxAtStartPar
(z.B. “MODEL 2602 (SN: 12345) @ COM1”)
oder “DUMMY” oder “” (leer, wenn nicht verbunden).

\end{description}


\sphinxlineitem{Return type}
\sphinxAtStartPar
str

\end{description}\end{quote}

\end{fulllineitems}

\index{is\_connected() (modules.smu.SmuManager.SmuManager method)@\spxentry{is\_connected()}\spxextra{modules.smu.SmuManager.SmuManager method}}

\begin{fulllineitems}
\phantomsection\label{\detokenize{modulab/smu_manager:modules.smu.SmuManager.SmuManager.is_connected}}
\pysigstartsignatures
\pysiglinewithargsret
{\sphinxbfcode{\sphinxupquote{is\_connected}}}
{}
{}
\pysigstopsignatures
\sphinxAtStartPar
Prüft, ob eine aktive und offene Verbindung zur SMU besteht.
\begin{quote}\begin{description}
\sphinxlineitem{Returns}
\sphinxAtStartPar
True, wenn verbunden und der Treiber als ‘offen’ gemeldet ist, sonst False.

\sphinxlineitem{Return type}
\sphinxAtStartPar
bool

\end{description}\end{quote}

\end{fulllineitems}

\index{reset\_channel() (modules.smu.SmuManager.SmuManager method)@\spxentry{reset\_channel()}\spxextra{modules.smu.SmuManager.SmuManager method}}

\begin{fulllineitems}
\phantomsection\label{\detokenize{modulab/smu_manager:modules.smu.SmuManager.SmuManager.reset_channel}}
\pysigstartsignatures
\pysiglinewithargsret
{\sphinxbfcode{\sphinxupquote{reset\_channel}}}
{\sphinxparam{\DUrole{n}{channel}}}
{}
\pysigstopsignatures
\sphinxAtStartPar
Setzt einen SMU\sphinxhyphen{}Kanal auf Werkseinstellungen zurück.
\begin{quote}\begin{description}
\sphinxlineitem{Parameters}
\sphinxAtStartPar
\sphinxstyleliteralstrong{\sphinxupquote{channel}} (\sphinxcode{\sphinxupquote{str}}) \textendash{} Der Kanal, der zurückgesetzt wird (z.B. ‘a’ or ‘b’).

\end{description}\end{quote}

\end{fulllineitems}

\index{set\_source\_voltage() (modules.smu.SmuManager.SmuManager method)@\spxentry{set\_source\_voltage()}\spxextra{modules.smu.SmuManager.SmuManager method}}

\begin{fulllineitems}
\phantomsection\label{\detokenize{modulab/smu_manager:modules.smu.SmuManager.SmuManager.set_source_voltage}}
\pysigstartsignatures
\pysiglinewithargsret
{\sphinxbfcode{\sphinxupquote{set\_source\_voltage}}}
{\sphinxparam{\DUrole{n}{channel}}}
{}
\pysigstopsignatures
\sphinxAtStartPar
Konfiguriert den Kanal als SPANNUNGSQUELLE (V\sphinxhyphen{}Source).

\sphinxAtStartPar
Aktualisiert auch den internen Zustand, damit \sphinxtitleref{set\_source\_limit}
und \sphinxtitleref{set\_source\_level} korrekt funktionieren.
\begin{quote}\begin{description}
\sphinxlineitem{Parameters}
\sphinxAtStartPar
\sphinxstyleliteralstrong{\sphinxupquote{channel}} (\sphinxcode{\sphinxupquote{str}}) \textendash{} Der zu konfigurierende Kanal (z.B. ‘a’).

\end{description}\end{quote}

\end{fulllineitems}

\index{set\_source\_current() (modules.smu.SmuManager.SmuManager method)@\spxentry{set\_source\_current()}\spxextra{modules.smu.SmuManager.SmuManager method}}

\begin{fulllineitems}
\phantomsection\label{\detokenize{modulab/smu_manager:modules.smu.SmuManager.SmuManager.set_source_current}}
\pysigstartsignatures
\pysiglinewithargsret
{\sphinxbfcode{\sphinxupquote{set\_source\_current}}}
{\sphinxparam{\DUrole{n}{channel}}}
{}
\pysigstopsignatures
\sphinxAtStartPar
Konfiguriert den Kanal als STROMQUELLE (I\sphinxhyphen{}Source).

\sphinxAtStartPar
Aktualisiert auch den internen Zustand, damit \sphinxtitleref{set\_source\_limit}
und \sphinxtitleref{set\_source\_level} korrekt funktionieren.
\begin{quote}\begin{description}
\sphinxlineitem{Parameters}
\sphinxAtStartPar
\sphinxstyleliteralstrong{\sphinxupquote{channel}} (\sphinxcode{\sphinxupquote{str}}) \textendash{} Der zu konfigurierende Kanal (z.B. ‘a’).

\end{description}\end{quote}

\end{fulllineitems}

\index{set\_sense\_local() (modules.smu.SmuManager.SmuManager method)@\spxentry{set\_sense\_local()}\spxextra{modules.smu.SmuManager.SmuManager method}}

\begin{fulllineitems}
\phantomsection\label{\detokenize{modulab/smu_manager:modules.smu.SmuManager.SmuManager.set_sense_local}}
\pysigstartsignatures
\pysiglinewithargsret
{\sphinxbfcode{\sphinxupquote{set\_sense\_local}}}
{\sphinxparam{\DUrole{n}{channel}}}
{}
\pysigstopsignatures
\sphinxAtStartPar
Stellt den Sense\sphinxhyphen{}Modus auf LOKAL (2\sphinxhyphen{}Draht\sphinxhyphen{}Messung).
\begin{quote}\begin{description}
\sphinxlineitem{Parameters}
\sphinxAtStartPar
\sphinxstyleliteralstrong{\sphinxupquote{channel}} (\sphinxcode{\sphinxupquote{str}}) \textendash{} Der zu konfigurierende Kanal (z.B. ‘a’).

\end{description}\end{quote}

\end{fulllineitems}

\index{set\_sense\_remote() (modules.smu.SmuManager.SmuManager method)@\spxentry{set\_sense\_remote()}\spxextra{modules.smu.SmuManager.SmuManager method}}

\begin{fulllineitems}
\phantomsection\label{\detokenize{modulab/smu_manager:modules.smu.SmuManager.SmuManager.set_sense_remote}}
\pysigstartsignatures
\pysiglinewithargsret
{\sphinxbfcode{\sphinxupquote{set\_sense\_remote}}}
{\sphinxparam{\DUrole{n}{channel}}}
{}
\pysigstopsignatures
\sphinxAtStartPar
Stellt den Sense\sphinxhyphen{}Modus auf REMOTE (4\sphinxhyphen{}Draht\sphinxhyphen{}Messung).
\begin{quote}\begin{description}
\sphinxlineitem{Parameters}
\sphinxAtStartPar
\sphinxstyleliteralstrong{\sphinxupquote{channel}} (\sphinxcode{\sphinxupquote{str}}) \textendash{} Der zu konfigurierende Kanal (z.B. ‘a’).

\end{description}\end{quote}

\end{fulllineitems}

\index{set\_source\_level() (modules.smu.SmuManager.SmuManager method)@\spxentry{set\_source\_level()}\spxextra{modules.smu.SmuManager.SmuManager method}}

\begin{fulllineitems}
\phantomsection\label{\detokenize{modulab/smu_manager:modules.smu.SmuManager.SmuManager.set_source_level}}
\pysigstartsignatures
\pysiglinewithargsret
{\sphinxbfcode{\sphinxupquote{set\_source\_level}}}
{\sphinxparam{\DUrole{n}{channel}}\sphinxparamcomma \sphinxparam{\DUrole{n}{level}}}
{}
\pysigstopsignatures
\sphinxAtStartPar
Setzt das Source\sphinxhyphen{}Level (V oder A).

\sphinxAtStartPar
Die Einheit (V oder A) hängt von der zuvor mit \sphinxtitleref{set\_source\_voltage}
oder \sphinxtitleref{set\_source\_current} konfigurierten Source\sphinxhyphen{}Funktion ab.
\begin{quote}\begin{description}
\sphinxlineitem{Parameters}\begin{itemize}
\item {} 
\sphinxAtStartPar
\sphinxstyleliteralstrong{\sphinxupquote{channel}} (\sphinxcode{\sphinxupquote{str}}) \textendash{} Der zu konfigurierende Kanal (z.B. ‘a’).

\item {} 
\sphinxAtStartPar
\sphinxstyleliteralstrong{\sphinxupquote{level}} (\sphinxcode{\sphinxupquote{float}}) \textendash{} Das zu setzende Level (in Volt oder Ampere).

\end{itemize}

\end{description}\end{quote}

\end{fulllineitems}

\index{set\_source\_limit() (modules.smu.SmuManager.SmuManager method)@\spxentry{set\_source\_limit()}\spxextra{modules.smu.SmuManager.SmuManager method}}

\begin{fulllineitems}
\phantomsection\label{\detokenize{modulab/smu_manager:modules.smu.SmuManager.SmuManager.set_source_limit}}
\pysigstartsignatures
\pysiglinewithargsret
{\sphinxbfcode{\sphinxupquote{set\_source\_limit}}}
{\sphinxparam{\DUrole{n}{channel}}\sphinxparamcomma \sphinxparam{\DUrole{n}{limit}}}
{}
\pysigstopsignatures
\sphinxAtStartPar
Setzt das Source\sphinxhyphen{}Limit (A oder V).

\sphinxAtStartPar
Die Einheit ist \sphinxstyleemphasis{entgegengesetzt} zur Source\sphinxhyphen{}Funktion:
\sphinxhyphen{} Wenn Source = Spannung (V), ist dies das Strom\sphinxhyphen{}Limit (A).
\sphinxhyphen{} Wenn Source = Strom (I), ist dies das Spannungs\sphinxhyphen{}Limit (V).
\begin{quote}\begin{description}
\sphinxlineitem{Parameters}\begin{itemize}
\item {} 
\sphinxAtStartPar
\sphinxstyleliteralstrong{\sphinxupquote{channel}} (\sphinxcode{\sphinxupquote{str}}) \textendash{} Der zu konfigurierende Kanal (z.B. ‘a’).

\item {} 
\sphinxAtStartPar
\sphinxstyleliteralstrong{\sphinxupquote{limit}} (\sphinxcode{\sphinxupquote{float}}) \textendash{} Das zu setzende Limit (in Ampere oder Volt).

\end{itemize}

\end{description}\end{quote}

\begin{sphinxadmonition}{note}{Examples}

\sphinxAtStartPar
Ein Strom\sphinxhyphen{}Limit (100mA) für eine Spannungsquelle setzen:

\begin{sphinxVerbatim}[commandchars=\\\{\}]
\PYG{c+c1}{\PYGZsh{} 1. Zuerst Kanal \PYGZsq{}a\PYGZsq{} als SPANNUNGSQUELLE definieren}
    \PYG{n}{manager}\PYG{o}{.}\PYG{n}{set\PYGZus{}source\PYGZus{}voltage}\PYG{p}{(}\PYG{l+s+s1}{\PYGZsq{}}\PYG{l+s+s1}{a}\PYG{l+s+s1}{\PYGZsq{}}\PYG{p}{)}

\PYG{c+c1}{\PYGZsh{} 2. Jetzt das Limit setzen (0.1 = 100mA STROM\PYGZhy{}Limit)}
\PYG{n}{manager}\PYG{o}{.}\PYG{n}{set\PYGZus{}source\PYGZus{}limit}\PYG{p}{(}\PYG{l+s+s1}{\PYGZsq{}}\PYG{l+s+s1}{a}\PYG{l+s+s1}{\PYGZsq{}}\PYG{p}{,} \PYG{l+m+mf}{0.1}\PYG{p}{)}
\end{sphinxVerbatim}

\sphinxAtStartPar
Ein Spannungs\sphinxhyphen{}Limit (20V) für eine Stromquelle setzen:

\begin{sphinxVerbatim}[commandchars=\\\{\}]
\PYG{c+c1}{\PYGZsh{} 1. Zuerst Kanal \PYGZsq{}b\PYGZsq{} als STROMQUELLE definieren}
\PYG{n}{manager}\PYG{o}{.}\PYG{n}{set\PYGZus{}source\PYGZus{}current}\PYG{p}{(}\PYG{l+s+s1}{\PYGZsq{}}\PYG{l+s+s1}{b}\PYG{l+s+s1}{\PYGZsq{}}\PYG{p}{)}

\PYG{c+c1}{\PYGZsh{} 2. Jetzt das Limit setzen (20.0 = 20V SPANNUNGS\PYGZhy{}Limit)}
\PYG{n}{manager}\PYG{o}{.}\PYG{n}{set\PYGZus{}source\PYGZus{}limit}\PYG{p}{(}\PYG{l+s+s1}{\PYGZsq{}}\PYG{l+s+s1}{b}\PYG{l+s+s1}{\PYGZsq{}}\PYG{p}{,} \PYG{l+m+mf}{20.0}\PYG{p}{)}
\end{sphinxVerbatim}
\end{sphinxadmonition}

\end{fulllineitems}

\index{set\_output\_state() (modules.smu.SmuManager.SmuManager method)@\spxentry{set\_output\_state()}\spxextra{modules.smu.SmuManager.SmuManager method}}

\begin{fulllineitems}
\phantomsection\label{\detokenize{modulab/smu_manager:modules.smu.SmuManager.SmuManager.set_output_state}}
\pysigstartsignatures
\pysiglinewithargsret
{\sphinxbfcode{\sphinxupquote{set\_output\_state}}}
{\sphinxparam{\DUrole{n}{channel}}\sphinxparamcomma \sphinxparam{\DUrole{n}{enable}}}
{}
\pysigstopsignatures
\sphinxAtStartPar
Schaltet den Ausgang eines Kanals EIN oder AUS.
\begin{quote}\begin{description}
\sphinxlineitem{Parameters}\begin{itemize}
\item {} 
\sphinxAtStartPar
\sphinxstyleliteralstrong{\sphinxupquote{channel}} (\sphinxcode{\sphinxupquote{str}}) \textendash{} Der zu schaltende Kanal (z.B. ‘a’).

\item {} 
\sphinxAtStartPar
\sphinxstyleliteralstrong{\sphinxupquote{enable}} (\sphinxcode{\sphinxupquote{bool}}) \textendash{} True, um den Ausgang einzuschalten, False, um ihn auszuschalten.

\end{itemize}

\end{description}\end{quote}

\end{fulllineitems}

\index{measure\_iv() (modules.smu.SmuManager.SmuManager method)@\spxentry{measure\_iv()}\spxextra{modules.smu.SmuManager.SmuManager method}}

\begin{fulllineitems}
\phantomsection\label{\detokenize{modulab/smu_manager:modules.smu.SmuManager.SmuManager.measure_iv}}
\pysigstartsignatures
\pysiglinewithargsret
{\sphinxbfcode{\sphinxupquote{measure\_iv}}}
{\sphinxparam{\DUrole{n}{channel}}}
{}
\pysigstopsignatures
\sphinxAtStartPar
Führt eine einzelne I/V\sphinxhyphen{}Messung auf dem Kanal durch.
… (andere Sektionen) …
\begin{quote}\begin{description}
\sphinxlineitem{Parameters}
\sphinxAtStartPar
\sphinxstyleliteralstrong{\sphinxupquote{channel}} (\sphinxcode{\sphinxupquote{str}}) \textendash{} Der zu messende Kanal (z.B. ‘a’).

\sphinxlineitem{Returns}
\sphinxAtStartPar
\begin{description}
\sphinxlineitem{Ein Tupel aus (Strom, Spannung) bei Erfolg.}
\sphinxAtStartPar
None bei einem Messfehler.

\end{description}


\sphinxlineitem{Return type}
\sphinxAtStartPar
tuple{[}float, float{]} | None

\end{description}\end{quote}

\begin{sphinxadmonition}{note}{Examples}

\sphinxAtStartPar
Das Auslesen der Messung und Speichern in Variablen:

\begin{sphinxVerbatim}[commandchars=\\\{\}]
\PYG{c+c1}{\PYGZsh{} Annahme: \PYGZsq{}manager\PYGZsq{} ist eine Instanz von SmuManager}
\PYG{n}{result} \PYG{o}{=} \PYG{n}{manager}\PYG{o}{.}\PYG{n}{measure\PYGZus{}iv}\PYG{p}{(}\PYG{l+s+s1}{\PYGZsq{}}\PYG{l+s+s1}{a}\PYG{l+s+s1}{\PYGZsq{}}\PYG{p}{)}

\PYG{k}{if} \PYG{n}{result}\PYG{p}{:}
    \PYG{n}{current}\PYG{p}{,} \PYG{n}{voltage} \PYG{o}{=} \PYG{n}{result}
    \PYG{n+nb}{print}\PYG{p}{(}\PYG{l+s+sa}{f}\PYG{l+s+s2}{\PYGZdq{}}\PYG{l+s+s2}{Messung OK: }\PYG{l+s+si}{\PYGZob{}}\PYG{n}{current}\PYG{l+s+si}{\PYGZcb{}}\PYG{l+s+s2}{ A, }\PYG{l+s+si}{\PYGZob{}}\PYG{n}{voltage}\PYG{l+s+si}{\PYGZcb{}}\PYG{l+s+s2}{ V}\PYG{l+s+s2}{\PYGZdq{}}\PYG{p}{)}
\PYG{k}{else}\PYG{p}{:}
    \PYG{n+nb}{print}\PYG{p}{(}\PYG{l+s+s2}{\PYGZdq{}}\PYG{l+s+s2}{Messung fehlgeschlagen oder keine Verbindung.}\PYG{l+s+s2}{\PYGZdq{}}\PYG{p}{)}
\end{sphinxVerbatim}
\end{sphinxadmonition}

\end{fulllineitems}


\end{fulllineitems}


\sphinxstepscope


\chapter{Spectrometer Manager}
\label{\detokenize{modulab/spectrometer_manager:module-modules.spectrometer.SpectrometerManager}}\label{\detokenize{modulab/spectrometer_manager:id1}}\label{\detokenize{modulab/spectrometer_manager:spectrometer-manager}}\label{\detokenize{modulab/spectrometer_manager::doc}}\index{module@\spxentry{module}!modules.spectrometer.SpectrometerManager@\spxentry{modules.spectrometer.SpectrometerManager}}\index{modules.spectrometer.SpectrometerManager@\spxentry{modules.spectrometer.SpectrometerManager}!module@\spxentry{module}}\index{SpectrometerManager (class in modules.spectrometer.SpectrometerManager)@\spxentry{SpectrometerManager}\spxextra{class in modules.spectrometer.SpectrometerManager}}

\begin{fulllineitems}
\phantomsection\label{\detokenize{modulab/spectrometer_manager:modules.spectrometer.SpectrometerManager.SpectrometerManager}}
\pysigstartsignatures
\pysiglinewithargsret
{\sphinxbfcode{\sphinxupquote{\DUrole{k}{class}\DUrole{w}{ }}}\sphinxcode{\sphinxupquote{modules.spectrometer.SpectrometerManager.}}\sphinxbfcode{\sphinxupquote{SpectrometerManager}}}
{\sphinxparam{\DUrole{n}{log\_manager}}\sphinxparamcomma \sphinxparam{\DUrole{n}{profile\_manager}}}
{}
\pysigstopsignatures
\sphinxAtStartPar
Bases: \sphinxcode{\sphinxupquote{QObject}}

\sphinxAtStartPar
Manager zur Steuerung und Verwaltung von Ocean Optics Spektrometern.

\sphinxAtStartPar
Diese Klasse kapselt die \sphinxtitleref{python\sphinxhyphen{}seabreeze}\sphinxhyphen{}Bibliothek, um eine stabile
Schnittstelle für die Geräteverbindung, Konfiguration (Integrationszeit,
Korrekturen) und Datenaufnahme (Spektren) bereitzustellen.
Sie nutzt PySide6\sphinxhyphen{}Signale, um die GUI über Statusänderungen zu informieren.
\begin{quote}\begin{description}
\sphinxlineitem{Parameters}\begin{itemize}
\item {} 
\sphinxAtStartPar
\sphinxstyleliteralstrong{\sphinxupquote{log\_manager}} (\sphinxcode{\sphinxupquote{LogManager}}) \textendash{} Eine Instanz eines Log\sphinxhyphen{}Managers (erwartet .info, .error, etc.).

\item {} 
\sphinxAtStartPar
\sphinxstyleliteralstrong{\sphinxupquote{profile\_manager}} (\sphinxcode{\sphinxupquote{ProfileManager}}) \textendash{} Eine Instanz zur Verwaltung von
App\sphinxhyphen{}Einstellungen (Lesen/Schreiben).

\end{itemize}

\end{description}\end{quote}
\begin{description}
\sphinxlineitem{Signale:}\begin{description}
\sphinxlineitem{connection\_status\_changed (bool, str):}
\sphinxAtStartPar
Wird ausgelöst, wenn sich der Verbindungsstatus ändert.
Args: (bool: verbunden, str: Gerätename/IDN).

\sphinxlineitem{device\_list\_updated (list):}
\sphinxAtStartPar
Wird ausgelöst, nachdem die Geräteliste aktualisiert wurde.
Args: (list: Liste von Gerätenamen {[}str{]}, z.B. {[}“FLAME (Q…)”, …{]}).

\sphinxlineitem{new\_spectrum\_acquired (numpy.ndarray, numpy.ndarray):}
\sphinxAtStartPar
Wird ausgelöst, wenn ein neues Spektrum verfügbar ist.
Args: (numpy.ndarray: Wellenlängen, numpy.ndarray: Intensitäten).

\end{description}

\end{description}
\index{\_\_init\_\_() (modules.spectrometer.SpectrometerManager.SpectrometerManager method)@\spxentry{\_\_init\_\_()}\spxextra{modules.spectrometer.SpectrometerManager.SpectrometerManager method}}

\begin{fulllineitems}
\phantomsection\label{\detokenize{modulab/spectrometer_manager:modules.spectrometer.SpectrometerManager.SpectrometerManager.__init__}}
\pysigstartsignatures
\pysiglinewithargsret
{\sphinxbfcode{\sphinxupquote{\_\_init\_\_}}}
{\sphinxparam{\DUrole{n}{log\_manager}}\sphinxparamcomma \sphinxparam{\DUrole{n}{profile\_manager}}}
{}
\pysigstopsignatures
\sphinxAtStartPar
Initialisiert den SpectrometerManager.

\sphinxAtStartPar
Lädt die zuletzt verwendete Konfiguration (Integrationszeit, Korrekturen)
aus dem ProfileManager und versucht automatisch, eine Verbindung
zum zuletzt verwendeten Gerät herzustellen.

\end{fulllineitems}

\index{get\_deviceList() (modules.spectrometer.SpectrometerManager.SpectrometerManager method)@\spxentry{get\_deviceList()}\spxextra{modules.spectrometer.SpectrometerManager.SpectrometerManager method}}

\begin{fulllineitems}
\phantomsection\label{\detokenize{modulab/spectrometer_manager:modules.spectrometer.SpectrometerManager.SpectrometerManager.get_deviceList}}
\pysigstartsignatures
\pysiglinewithargsret
{\sphinxbfcode{\sphinxupquote{get\_deviceList}}}
{}
{}
\pysigstopsignatures
\sphinxAtStartPar
Scannt nach verfügbaren Spektrometern und aktualisiert die interne Liste.

\sphinxAtStartPar
Verwendet \sphinxtitleref{seabreeze.list\_devices()} und erstellt eine Zuordnung (Map)
von formatierten Gerätenamen zu Geräteinstanzen. Löst das
\sphinxtitleref{device\_list\_updated}\sphinxhyphen{}Signal aus.
\begin{quote}\begin{description}
\sphinxlineitem{Returns}
\sphinxAtStartPar
Eine Liste formatierter Gerätenamen (z.B. {[}“Oceanoptics (O…)”, “USB2000 (…)”{]}).

\sphinxlineitem{Return type}
\sphinxAtStartPar
list

\end{description}\end{quote}

\begin{sphinxadmonition}{note}{Examples}

\sphinxAtStartPar
Eine Geräteliste abrufen und in einer Combobox anzeigen:

\begin{sphinxVerbatim}[commandchars=\\\{\}]
\PYG{c+c1}{\PYGZsh{} Annahme: \PYGZsq{}spectrometer\PYGZus{}mgr\PYGZsq{} ist eine Instanz von SpectrometerManager}
\PYG{c+c1}{\PYGZsh{} und \PYGZsq{}ui.combo\PYGZus{}devices\PYGZsq{} ist eine QComboBox.}

\PYG{c+c1}{\PYGZsh{} Zuerst das Signal verbinden (z.B. in \PYGZus{}\PYGZus{}init\PYGZus{}\PYGZus{} der GUI)}
\PYG{n}{spectrometer\PYGZus{}mgr}\PYG{o}{.}\PYG{n}{device\PYGZus{}list\PYGZus{}updated}\PYG{o}{.}\PYG{n}{connect}\PYG{p}{(}
    \PYG{k}{lambda} \PYG{n}{devices}\PYG{p}{:} \PYG{n}{ui}\PYG{o}{.}\PYG{n}{combo\PYGZus{}devices}\PYG{o}{.}\PYG{n}{addItems}\PYG{p}{(}\PYG{n}{devices}\PYG{p}{)}
\PYG{p}{)}

\PYG{c+c1}{\PYGZsh{} Manuell eine Aktualisierung auslösen}
\PYG{n}{ui}\PYG{o}{.}\PYG{n}{combo\PYGZus{}devices}\PYG{o}{.}\PYG{n}{clear}\PYG{p}{(}\PYG{p}{)}
\PYG{n}{spectrometer\PYGZus{}mgr}\PYG{o}{.}\PYG{n}{get\PYGZus{}deviceList}\PYG{p}{(}\PYG{p}{)}
\end{sphinxVerbatim}
\end{sphinxadmonition}

\end{fulllineitems}

\index{connect() (modules.spectrometer.SpectrometerManager.SpectrometerManager method)@\spxentry{connect()}\spxextra{modules.spectrometer.SpectrometerManager.SpectrometerManager method}}

\begin{fulllineitems}
\phantomsection\label{\detokenize{modulab/spectrometer_manager:modules.spectrometer.SpectrometerManager.SpectrometerManager.connect}}
\pysigstartsignatures
\pysiglinewithargsret
{\sphinxbfcode{\sphinxupquote{connect}}}
{\sphinxparam{\DUrole{n}{device\_name\_or\_serial}}}
{}
\pysigstopsignatures
\sphinxAtStartPar
Verbindet ein Spektrometer über seinen Namen oder seine Seriennummer.

\sphinxAtStartPar
Trennt zuerst eine eventuell bestehende Verbindung.
Speichert die Seriennummer bei Erfolg für die Wiederverbindung.
\begin{quote}\begin{description}
\sphinxlineitem{Parameters}
\sphinxAtStartPar
\sphinxstyleliteralstrong{\sphinxupquote{device\_name\_or\_serial}} (\sphinxcode{\sphinxupquote{str}}) \textendash{} Kann entweder der formatierte Name aus \sphinxtitleref{get\_deviceList()}
(z.B. “FLAME (Q…)”) oder die reine Seriennummer
(z.B. “Q…”) sein.

\sphinxlineitem{Returns}
\sphinxAtStartPar
True bei erfolgreicher Verbindung, sonst False.

\sphinxlineitem{Return type}
\sphinxAtStartPar
bool

\end{description}\end{quote}

\begin{sphinxadmonition}{note}{Examples}

\sphinxAtStartPar
Verbindung über den formatierten Namen (z.B. aus einer Combobox):

\begin{sphinxVerbatim}[commandchars=\\\{\}]
\PYG{n}{name} \PYG{o}{=} \PYG{l+s+s2}{\PYGZdq{}}\PYG{l+s+s2}{FLAME (QEP20488)}\PYG{l+s+s2}{\PYGZdq{}}
\PYG{n}{success} \PYG{o}{=} \PYG{n}{spectrometer\PYGZus{}mgr}\PYG{o}{.}\PYG{n}{connect}\PYG{p}{(}\PYG{n}{name}\PYG{p}{)}
\PYG{k}{if} \PYG{n}{success}\PYG{p}{:}
    \PYG{n+nb}{print}\PYG{p}{(}\PYG{l+s+s2}{\PYGZdq{}}\PYG{l+s+s2}{Verbunden mit}\PYG{l+s+s2}{\PYGZdq{}}\PYG{p}{,} \PYG{n}{name}\PYG{p}{)}
\end{sphinxVerbatim}

\sphinxAtStartPar
Verbindung direkt über die Seriennummer:

\begin{sphinxVerbatim}[commandchars=\\\{\}]
\PYG{n}{serial} \PYG{o}{=} \PYG{l+s+s2}{\PYGZdq{}}\PYG{l+s+s2}{QEP20488}\PYG{l+s+s2}{\PYGZdq{}}
\PYG{n}{spectrometer\PYGZus{}mgr}\PYG{o}{.}\PYG{n}{connect}\PYG{p}{(}\PYG{n}{serial}\PYG{p}{)}
\end{sphinxVerbatim}
\end{sphinxadmonition}

\end{fulllineitems}

\index{connect\_LastDevice() (modules.spectrometer.SpectrometerManager.SpectrometerManager method)@\spxentry{connect\_LastDevice()}\spxextra{modules.spectrometer.SpectrometerManager.SpectrometerManager method}}

\begin{fulllineitems}
\phantomsection\label{\detokenize{modulab/spectrometer_manager:modules.spectrometer.SpectrometerManager.SpectrometerManager.connect_LastDevice}}
\pysigstartsignatures
\pysiglinewithargsret
{\sphinxbfcode{\sphinxupquote{connect\_LastDevice}}}
{}
{}
\pysigstopsignatures
\sphinxAtStartPar
Versucht, die Verbindung mit dem zuletzt genutzten Gerät wiederherzustellen.

\sphinxAtStartPar
Verwendet die im Profil gespeicherte Seriennummer (\sphinxtitleref{self.LastDevice}).
\begin{quote}\begin{description}
\sphinxlineitem{Returns}
\sphinxAtStartPar
\begin{description}
\sphinxlineitem{True bei Erfolg, False, wenn kein Gerät gespeichert war}
\sphinxAtStartPar
oder die Verbindung fehlschlägt.

\end{description}


\sphinxlineitem{Return type}
\sphinxAtStartPar
bool

\end{description}\end{quote}

\end{fulllineitems}

\index{disconnect() (modules.spectrometer.SpectrometerManager.SpectrometerManager method)@\spxentry{disconnect()}\spxextra{modules.spectrometer.SpectrometerManager.SpectrometerManager method}}

\begin{fulllineitems}
\phantomsection\label{\detokenize{modulab/spectrometer_manager:modules.spectrometer.SpectrometerManager.SpectrometerManager.disconnect}}
\pysigstartsignatures
\pysiglinewithargsret
{\sphinxbfcode{\sphinxupquote{disconnect}}}
{}
{}
\pysigstopsignatures
\sphinxAtStartPar
Trennt die aktive Verbindung zum Spektrometer.

\sphinxAtStartPar
Schließt das Gerät über \sphinxtitleref{spectrometer.close()} und löst
\sphinxtitleref{connection\_status\_changed} aus.

\end{fulllineitems}

\index{get\_activeDeviceName() (modules.spectrometer.SpectrometerManager.SpectrometerManager method)@\spxentry{get\_activeDeviceName()}\spxextra{modules.spectrometer.SpectrometerManager.SpectrometerManager method}}

\begin{fulllineitems}
\phantomsection\label{\detokenize{modulab/spectrometer_manager:modules.spectrometer.SpectrometerManager.SpectrometerManager.get_activeDeviceName}}
\pysigstartsignatures
\pysiglinewithargsret
{\sphinxbfcode{\sphinxupquote{get\_activeDeviceName}}}
{}
{}
\pysigstopsignatures
\sphinxAtStartPar
Gibt den formatierten Namen des aktuell verbundenen Geräts zurück.
\begin{quote}\begin{description}
\sphinxlineitem{Returns}
\sphinxAtStartPar
\begin{description}
\sphinxlineitem{Der formatierte Gerätename (z.B. “FLAME (Q…)”)}
\sphinxAtStartPar
oder “” (leer, wenn nicht verbunden).

\end{description}


\sphinxlineitem{Return type}
\sphinxAtStartPar
str

\end{description}\end{quote}

\end{fulllineitems}

\index{is\_connected() (modules.spectrometer.SpectrometerManager.SpectrometerManager method)@\spxentry{is\_connected()}\spxextra{modules.spectrometer.SpectrometerManager.SpectrometerManager method}}

\begin{fulllineitems}
\phantomsection\label{\detokenize{modulab/spectrometer_manager:modules.spectrometer.SpectrometerManager.SpectrometerManager.is_connected}}
\pysigstartsignatures
\pysiglinewithargsret
{\sphinxbfcode{\sphinxupquote{is\_connected}}}
{}
{}
\pysigstopsignatures
\sphinxAtStartPar
Prüft, ob eine aktive Verbindung zum Spektrometer besteht.
\begin{quote}\begin{description}
\sphinxlineitem{Returns}
\sphinxAtStartPar
True, wenn verbunden, sonst False.

\sphinxlineitem{Return type}
\sphinxAtStartPar
bool

\end{description}\end{quote}

\end{fulllineitems}

\index{set\_correction\_dark\_count() (modules.spectrometer.SpectrometerManager.SpectrometerManager method)@\spxentry{set\_correction\_dark\_count()}\spxextra{modules.spectrometer.SpectrometerManager.SpectrometerManager method}}

\begin{fulllineitems}
\phantomsection\label{\detokenize{modulab/spectrometer_manager:modules.spectrometer.SpectrometerManager.SpectrometerManager.set_correction_dark_count}}
\pysigstartsignatures
\pysiglinewithargsret
{\sphinxbfcode{\sphinxupquote{set\_correction\_dark\_count}}}
{\sphinxparam{\DUrole{n}{enable}}}
{}
\pysigstopsignatures
\sphinxAtStartPar
Aktiviert/Deaktiviert die Korrektur des Dunkelstroms (Dark Counts).

\sphinxAtStartPar
Wenn aktiviert, wird der Durchschnittswert der elektrisch verdunkelten
Pixel vom Spektrum abgezogen.
\begin{quote}\begin{description}
\sphinxlineitem{Parameters}
\sphinxAtStartPar
\sphinxstyleliteralstrong{\sphinxupquote{enable}} (\sphinxcode{\sphinxupquote{bool}}) \textendash{} True, um die Korrektur zu aktivieren, False zum Deaktivieren.

\end{description}\end{quote}

\begin{sphinxadmonition}{note}{Examples}

\sphinxAtStartPar
Dunkelstrom\sphinxhyphen{}Korrektur aktivieren:

\begin{sphinxVerbatim}[commandchars=\\\{\}]
\PYG{n}{spectrometer\PYGZus{}mgr}\PYG{o}{.}\PYG{n}{set\PYGZus{}correction\PYGZus{}dark\PYGZus{}count}\PYG{p}{(}\PYG{k+kc}{True}\PYG{p}{)}
\end{sphinxVerbatim}
\end{sphinxadmonition}

\end{fulllineitems}

\index{get\_correction\_dark\_count() (modules.spectrometer.SpectrometerManager.SpectrometerManager method)@\spxentry{get\_correction\_dark\_count()}\spxextra{modules.spectrometer.SpectrometerManager.SpectrometerManager method}}

\begin{fulllineitems}
\phantomsection\label{\detokenize{modulab/spectrometer_manager:modules.spectrometer.SpectrometerManager.SpectrometerManager.get_correction_dark_count}}
\pysigstartsignatures
\pysiglinewithargsret
{\sphinxbfcode{\sphinxupquote{get\_correction\_dark\_count}}}
{}
{}
\pysigstopsignatures
\sphinxAtStartPar
Gibt den aktuellen Status der Dunkelstrom\sphinxhyphen{}Korrektur zurück.
\begin{quote}\begin{description}
\sphinxlineitem{Returns}
\sphinxAtStartPar
True, wenn die Korrektur aktiv ist.

\sphinxlineitem{Return type}
\sphinxAtStartPar
bool

\end{description}\end{quote}

\end{fulllineitems}

\index{set\_correction\_non\_linearity() (modules.spectrometer.SpectrometerManager.SpectrometerManager method)@\spxentry{set\_correction\_non\_linearity()}\spxextra{modules.spectrometer.SpectrometerManager.SpectrometerManager method}}

\begin{fulllineitems}
\phantomsection\label{\detokenize{modulab/spectrometer_manager:modules.spectrometer.SpectrometerManager.SpectrometerManager.set_correction_non_linearity}}
\pysigstartsignatures
\pysiglinewithargsret
{\sphinxbfcode{\sphinxupquote{set\_correction\_non\_linearity}}}
{\sphinxparam{\DUrole{n}{enable}}}
{}
\pysigstopsignatures
\sphinxAtStartPar
Aktiviert/Deaktiviert die Nichtlinearitäts\sphinxhyphen{}Korrektur.

\sphinxAtStartPar
Wenn aktiviert und vom Gerät unterstützt, werden die Messwerte
anhand der im EEPROM gespeicherten Koeffizienten linearisiert.
\begin{quote}\begin{description}
\sphinxlineitem{Parameters}
\sphinxAtStartPar
\sphinxstyleliteralstrong{\sphinxupquote{enable}} (\sphinxcode{\sphinxupquote{bool}}) \textendash{} True, um die Korrektur zu aktivieren, False zum Deaktivieren.

\end{description}\end{quote}

\begin{sphinxadmonition}{note}{Examples}

\sphinxAtStartPar
Nichtlinearitäts\sphinxhyphen{}Korrektur deaktivieren:

\begin{sphinxVerbatim}[commandchars=\\\{\}]
\PYG{n}{spectrometer\PYGZus{}mgr}\PYG{o}{.}\PYG{n}{set\PYGZus{}correction\PYGZus{}non\PYGZus{}linearity}\PYG{p}{(}\PYG{k+kc}{False}\PYG{p}{)}
\end{sphinxVerbatim}
\end{sphinxadmonition}

\end{fulllineitems}

\index{get\_correction\_non\_linearity() (modules.spectrometer.SpectrometerManager.SpectrometerManager method)@\spxentry{get\_correction\_non\_linearity()}\spxextra{modules.spectrometer.SpectrometerManager.SpectrometerManager method}}

\begin{fulllineitems}
\phantomsection\label{\detokenize{modulab/spectrometer_manager:modules.spectrometer.SpectrometerManager.SpectrometerManager.get_correction_non_linearity}}
\pysigstartsignatures
\pysiglinewithargsret
{\sphinxbfcode{\sphinxupquote{get\_correction\_non\_linearity}}}
{}
{}
\pysigstopsignatures
\sphinxAtStartPar
Gibt den aktuellen Status der Nichtlinearitäts\sphinxhyphen{}Korrektur zurück.
\begin{quote}\begin{description}
\sphinxlineitem{Returns}
\sphinxAtStartPar
True, wenn die Korrektur aktiv ist.

\sphinxlineitem{Return type}
\sphinxAtStartPar
bool

\end{description}\end{quote}

\end{fulllineitems}

\index{set\_integrationtime() (modules.spectrometer.SpectrometerManager.SpectrometerManager method)@\spxentry{set\_integrationtime()}\spxextra{modules.spectrometer.SpectrometerManager.SpectrometerManager method}}

\begin{fulllineitems}
\phantomsection\label{\detokenize{modulab/spectrometer_manager:modules.spectrometer.SpectrometerManager.SpectrometerManager.set_integrationtime}}
\pysigstartsignatures
\pysiglinewithargsret
{\sphinxbfcode{\sphinxupquote{set\_integrationtime}}}
{\sphinxparam{\DUrole{n}{time\_us}}}
{}
\pysigstopsignatures
\sphinxAtStartPar
Stellt die Integrationszeit des Spektrometers in Mikrosekunden (us) ein.

\sphinxAtStartPar
Die Zeit wird automatisch auf die vom Gerät unterstützten Hardware\sphinxhyphen{}Limits
begrenzt (clamping). Die Einstellung wird auch im Profil gespeichert.
\begin{quote}\begin{description}
\sphinxlineitem{Parameters}
\sphinxAtStartPar
\sphinxstyleliteralstrong{\sphinxupquote{time\_us}} (\sphinxcode{\sphinxupquote{int}}) \textendash{} Die gewünschte Integrationszeit in Mikrosekunden.

\sphinxlineitem{Returns}
\sphinxAtStartPar
\begin{description}
\sphinxlineitem{True, wenn die Zeit erfolgreich gesetzt (oder zwischengespeichert)}
\sphinxAtStartPar
wurde, False bei einem Hardware\sphinxhyphen{}Fehler.

\end{description}


\sphinxlineitem{Return type}
\sphinxAtStartPar
bool

\end{description}\end{quote}

\begin{sphinxadmonition}{note}{Examples}

\sphinxAtStartPar
Integrationszeit auf 100 Millisekunden (100.000 µs) setzen:

\begin{sphinxVerbatim}[commandchars=\\\{\}]
\PYG{n}{spectrometer\PYGZus{}mgr}\PYG{o}{.}\PYG{n}{set\PYGZus{}integrationtime}\PYG{p}{(}\PYG{l+m+mi}{100} \PYG{o}{*} \PYG{l+m+mi}{1000}\PYG{p}{)}
\end{sphinxVerbatim}

\sphinxAtStartPar
Integrationszeit auf 2 Sekunden (2.000.000 µs) setzen:

\begin{sphinxVerbatim}[commandchars=\\\{\}]
\PYG{n}{spectrometer\PYGZus{}mgr}\PYG{o}{.}\PYG{n}{set\PYGZus{}integrationtime}\PYG{p}{(}\PYG{l+m+mi}{2\PYGZus{}000\PYGZus{}000}\PYG{p}{)}
\end{sphinxVerbatim}
\end{sphinxadmonition}

\end{fulllineitems}

\index{get\_integrationtime() (modules.spectrometer.SpectrometerManager.SpectrometerManager method)@\spxentry{get\_integrationtime()}\spxextra{modules.spectrometer.SpectrometerManager.SpectrometerManager method}}

\begin{fulllineitems}
\phantomsection\label{\detokenize{modulab/spectrometer_manager:modules.spectrometer.SpectrometerManager.SpectrometerManager.get_integrationtime}}
\pysigstartsignatures
\pysiglinewithargsret
{\sphinxbfcode{\sphinxupquote{get\_integrationtime}}}
{}
{}
\pysigstopsignatures
\sphinxAtStartPar
Gibt die zuletzt erfolgreich gesetzte Integrationszeit in Mikrosekunden (us) zurück.
\begin{quote}\begin{description}
\sphinxlineitem{Returns}
\sphinxAtStartPar
Die Integrationszeit in Mikrosekunden.

\sphinxlineitem{Return type}
\sphinxAtStartPar
int

\end{description}\end{quote}

\end{fulllineitems}

\index{get\_integrationtime\_limits\_us() (modules.spectrometer.SpectrometerManager.SpectrometerManager method)@\spxentry{get\_integrationtime\_limits\_us()}\spxextra{modules.spectrometer.SpectrometerManager.SpectrometerManager method}}

\begin{fulllineitems}
\phantomsection\label{\detokenize{modulab/spectrometer_manager:modules.spectrometer.SpectrometerManager.SpectrometerManager.get_integrationtime_limits_us}}
\pysigstartsignatures
\pysiglinewithargsret
{\sphinxbfcode{\sphinxupquote{get\_integrationtime\_limits\_us}}}
{}
{}
\pysigstopsignatures
\sphinxAtStartPar
Gibt die Hardware\sphinxhyphen{}Limits (min, max) der Integrationszeit in Mikrosekunden zurück.
\begin{quote}\begin{description}
\sphinxlineitem{Returns}
\sphinxAtStartPar
\begin{description}
\sphinxlineitem{(min\_integrationszeit\_us, max\_integrationszeit\_us).}
\sphinxAtStartPar
Gibt (0, 0) zurück, wenn nicht verbunden.

\end{description}


\sphinxlineitem{Return type}
\sphinxAtStartPar
tuple{[}int, int{]}

\end{description}\end{quote}

\begin{sphinxadmonition}{note}{Examples}

\sphinxAtStartPar
Minimale und maximale Zeit abfragen:

\begin{sphinxVerbatim}[commandchars=\\\{\}]
\PYG{n}{min\PYGZus{}t}\PYG{p}{,} \PYG{n}{max\PYGZus{}t} \PYG{o}{=} \PYG{n}{spectrometer\PYGZus{}mgr}\PYG{o}{.}\PYG{n}{get\PYGZus{}integrationtime\PYGZus{}limits\PYGZus{}us}\PYG{p}{(}\PYG{p}{)}
\PYG{k}{if} \PYG{n}{max\PYGZus{}t} \PYG{o}{\PYGZgt{}} \PYG{l+m+mi}{0}\PYG{p}{:}
    \PYG{n+nb}{print}\PYG{p}{(}\PYG{l+s+sa}{f}\PYG{l+s+s2}{\PYGZdq{}}\PYG{l+s+s2}{Unterstützter Bereich: }\PYG{l+s+si}{\PYGZob{}}\PYG{n}{min\PYGZus{}t}\PYG{l+s+si}{\PYGZcb{}}\PYG{l+s+s2}{ µs bis }\PYG{l+s+si}{\PYGZob{}}\PYG{n}{max\PYGZus{}t}\PYG{l+s+si}{\PYGZcb{}}\PYG{l+s+s2}{ µs}\PYG{l+s+s2}{\PYGZdq{}}\PYG{p}{)}
\end{sphinxVerbatim}
\end{sphinxadmonition}

\end{fulllineitems}

\index{get\_max\_intensity() (modules.spectrometer.SpectrometerManager.SpectrometerManager method)@\spxentry{get\_max\_intensity()}\spxextra{modules.spectrometer.SpectrometerManager.SpectrometerManager method}}

\begin{fulllineitems}
\phantomsection\label{\detokenize{modulab/spectrometer_manager:modules.spectrometer.SpectrometerManager.SpectrometerManager.get_max_intensity}}
\pysigstartsignatures
\pysiglinewithargsret
{\sphinxbfcode{\sphinxupquote{get\_max\_intensity}}}
{}
{}
\pysigstopsignatures
\sphinxAtStartPar
Gibt die maximal mögliche Intensität (ADC\sphinxhyphen{}Sättigungswert) des Spektrometers zurück.

\sphinxAtStartPar
Dies ist typischerweise 65535.0 (für 16\sphinxhyphen{}bit ADC) oder 4095.0 (für 12\sphinxhyphen{}bit ADC).
\begin{quote}\begin{description}
\sphinxlineitem{Returns}
\sphinxAtStartPar
\begin{description}
\sphinxlineitem{Der maximale Sättigungswert.}
\sphinxAtStartPar
Gibt 65535.0 als Fallback zurück, wenn nicht verbunden.

\end{description}


\sphinxlineitem{Return type}
\sphinxAtStartPar
float

\end{description}\end{quote}

\end{fulllineitems}

\index{acquire\_spectrum() (modules.spectrometer.SpectrometerManager.SpectrometerManager method)@\spxentry{acquire\_spectrum()}\spxextra{modules.spectrometer.SpectrometerManager.SpectrometerManager method}}

\begin{fulllineitems}
\phantomsection\label{\detokenize{modulab/spectrometer_manager:modules.spectrometer.SpectrometerManager.SpectrometerManager.acquire_spectrum}}
\pysigstartsignatures
\pysiglinewithargsret
{\sphinxbfcode{\sphinxupquote{acquire\_spectrum}}}
{}
{}
\pysigstopsignatures
\sphinxAtStartPar
Nimmt ein einzelnes Spektrum mit den aktuell gesetzten Korrekturen auf.

\sphinxAtStartPar
Löst bei Erfolg das \sphinxtitleref{new\_spectrum\_acquired}\sphinxhyphen{}Signal aus.
\begin{quote}\begin{description}
\sphinxlineitem{Returns}
\sphinxAtStartPar
Ein Tupel aus (wavelengths, intensities) bei Erfolg.
(None, None) bei einem Messfehler oder wenn nicht verbunden.

\sphinxlineitem{Return type}
\sphinxAtStartPar
tuple{[}np.ndarray | None, np.ndarray | None{]}

\end{description}\end{quote}

\begin{sphinxadmonition}{note}{Examples}

\sphinxAtStartPar
Ein Spektrum aufnehmen und verarbeiten:

\begin{sphinxVerbatim}[commandchars=\\\{\}]
\PYG{c+c1}{\PYGZsh{} Annahme: \PYGZsq{}spectrometer\PYGZus{}mgr\PYGZsq{} ist eine Instanz von SpectrometerManager}

\PYG{n}{wavelengths}\PYG{p}{,} \PYG{n}{intensities} \PYG{o}{=} \PYG{n}{spectrometer\PYGZus{}mgr}\PYG{o}{.}\PYG{n}{acquire\PYGZus{}spectrum}\PYG{p}{(}\PYG{p}{)}

\PYG{k}{if} \PYG{n}{wavelengths} \PYG{o+ow}{is} \PYG{o+ow}{not} \PYG{k+kc}{None}\PYG{p}{:}
    \PYG{c+c1}{\PYGZsh{} Finde die Wellenlänge mit der maximalen Intensität}
    \PYG{n}{peak\PYGZus{}index} \PYG{o}{=} \PYG{n}{np}\PYG{o}{.}\PYG{n}{argmax}\PYG{p}{(}\PYG{n}{intensities}\PYG{p}{)}
    \PYG{n}{peak\PYGZus{}wl} \PYG{o}{=} \PYG{n}{wavelengths}\PYG{p}{[}\PYG{n}{peak\PYGZus{}index}\PYG{p}{]}
    \PYG{n}{peak\PYGZus{}int} \PYG{o}{=} \PYG{n}{intensities}\PYG{p}{[}\PYG{n}{peak\PYGZus{}index}\PYG{p}{]}

    \PYG{n+nb}{print}\PYG{p}{(}\PYG{l+s+sa}{f}\PYG{l+s+s2}{\PYGZdq{}}\PYG{l+s+s2}{Stärkstes Signal bei }\PYG{l+s+si}{\PYGZob{}}\PYG{n}{peak\PYGZus{}wl}\PYG{l+s+si}{:}\PYG{l+s+s2}{.2f}\PYG{l+s+si}{\PYGZcb{}}\PYG{l+s+s2}{ nm mit }\PYG{l+s+si}{\PYGZob{}}\PYG{n}{peak\PYGZus{}int}\PYG{l+s+si}{:}\PYG{l+s+s2}{.0f}\PYG{l+s+si}{\PYGZcb{}}\PYG{l+s+s2}{ Counts}\PYG{l+s+s2}{\PYGZdq{}}\PYG{p}{)}
\PYG{k}{else}\PYG{p}{:}
    \PYG{n+nb}{print}\PYG{p}{(}\PYG{l+s+s2}{\PYGZdq{}}\PYG{l+s+s2}{Spektrum\PYGZhy{}Aufnahme fehlgeschlagen.}\PYG{l+s+s2}{\PYGZdq{}}\PYG{p}{)}
\end{sphinxVerbatim}
\end{sphinxadmonition}

\end{fulllineitems}


\end{fulllineitems}


\sphinxstepscope


\chapter{Export Manager}
\label{\detokenize{modulab/export_manager:module-modules.export.ExportManager}}\label{\detokenize{modulab/export_manager:id1}}\label{\detokenize{modulab/export_manager:export-manager}}\label{\detokenize{modulab/export_manager::doc}}\index{module@\spxentry{module}!modules.export.ExportManager@\spxentry{modules.export.ExportManager}}\index{modules.export.ExportManager@\spxentry{modules.export.ExportManager}!module@\spxentry{module}}\index{ExportManager (class in modules.export.ExportManager)@\spxentry{ExportManager}\spxextra{class in modules.export.ExportManager}}

\begin{fulllineitems}
\phantomsection\label{\detokenize{modulab/export_manager:modules.export.ExportManager.ExportManager}}
\pysigstartsignatures
\pysiglinewithargsret
{\sphinxbfcode{\sphinxupquote{\DUrole{k}{class}\DUrole{w}{ }}}\sphinxcode{\sphinxupquote{modules.export.ExportManager.}}\sphinxbfcode{\sphinxupquote{ExportManager}}}
{\sphinxparam{\DUrole{n}{log\_manager}}\sphinxparamcomma \sphinxparam{\DUrole{n}{profile\_manager}}}
{}
\pysigstopsignatures
\sphinxAtStartPar
Bases: \sphinxcode{\sphinxupquote{QObject}}

\sphinxAtStartPar
Verwaltet den Daten\sphinxhyphen{}Export in HDF5\sphinxhyphen{}Dateien und fungiert als Datenquelle für Live\sphinxhyphen{}Plots.

\sphinxAtStartPar
Dieser Manager abstrahiert die Komplexität von \sphinxtitleref{h5py}. Er implementiert ein
Zeilen\sphinxhyphen{}basiertes Schreibmodell: Daten werden mit \sphinxtitleref{add()} gesammelt (gestaged)
und mit \sphinxtitleref{commit()} synchron in die HDF5\sphinxhyphen{}Datei geschrieben und gleichzeitig
an die GUI (z.B. PlotManager) gesendet.
\begin{description}
\sphinxlineitem{Funktionsweise:}\begin{enumerate}
\sphinxsetlistlabels{\arabic}{enumi}{enumii}{}{.}%
\item {} 
\sphinxAtStartPar
\sphinxstylestrong{Setup:} Zielordner wählen (\sphinxtitleref{select\_directory\_dialog}).

\item {} 
\sphinxAtStartPar
\sphinxstylestrong{Start:} Neue Datei/Gruppe erstellen (\sphinxtitleref{new}).

\item {} 
\sphinxAtStartPar
\sphinxstylestrong{Metadaten:} Statische Infos speichern (\sphinxtitleref{add\_static}).

\item {} \begin{description}
\sphinxlineitem{\sphinxstylestrong{Loop:}}\begin{itemize}
\item {} 
\sphinxAtStartPar
Werte hinzufügen (\sphinxtitleref{add(‘Voltage’, 5.0)}).

\item {} 
\sphinxAtStartPar
Werte hinzufügen (\sphinxtitleref{add(‘Current’, 0.001)}).

\item {} 
\sphinxAtStartPar
Schreiben \& Senden (\sphinxtitleref{commit()}).

\end{itemize}

\end{description}

\item {} 
\sphinxAtStartPar
\sphinxstylestrong{Ende:} Datei schließen (\sphinxtitleref{stop}).

\end{enumerate}

\end{description}
\begin{quote}\begin{description}
\sphinxlineitem{Parameters}\begin{itemize}
\item {} 
\sphinxAtStartPar
\sphinxstyleliteralstrong{\sphinxupquote{log\_manager}} (\sphinxcode{\sphinxupquote{LogManager}}) \textendash{} Instanz für das Logging.

\item {} 
\sphinxAtStartPar
\sphinxstyleliteralstrong{\sphinxupquote{profile\_manager}} (\sphinxcode{\sphinxupquote{ProfileManager}}) \textendash{} Instanz zum Speichern des letzten Pfades.

\end{itemize}

\end{description}\end{quote}
\begin{description}
\sphinxlineitem{Signale:}\begin{description}
\sphinxlineitem{data\_committed (dict):}
\sphinxAtStartPar
Wird bei jedem \sphinxtitleref{commit()} ausgelöst. Enthält die neuen Datenpunkte für
Live\sphinxhyphen{}Plots.

\sphinxAtStartPar
\sphinxstylestrong{Payload\sphinxhyphen{}Struktur:}

\begin{sphinxVerbatim}[commandchars=\\\{\}]
\PYG{p}{\PYGZob{}}
    \PYG{l+s+s1}{\PYGZsq{}}\PYG{l+s+s1}{Voltage}\PYG{l+s+s1}{\PYGZsq{}}\PYG{p}{:} \PYG{p}{\PYGZob{}}\PYG{l+s+s1}{\PYGZsq{}}\PYG{l+s+s1}{value}\PYG{l+s+s1}{\PYGZsq{}}\PYG{p}{:} \PYG{l+m+mf}{5.0}\PYG{p}{,} \PYG{l+s+s1}{\PYGZsq{}}\PYG{l+s+s1}{unit}\PYG{l+s+s1}{\PYGZsq{}}\PYG{p}{:} \PYG{l+s+s1}{\PYGZsq{}}\PYG{l+s+s1}{V}\PYG{l+s+s1}{\PYGZsq{}}\PYG{p}{\PYGZcb{}}\PYG{p}{,}
    \PYG{l+s+s1}{\PYGZsq{}}\PYG{l+s+s1}{Current}\PYG{l+s+s1}{\PYGZsq{}}\PYG{p}{:} \PYG{p}{\PYGZob{}}\PYG{l+s+s1}{\PYGZsq{}}\PYG{l+s+s1}{value}\PYG{l+s+s1}{\PYGZsq{}}\PYG{p}{:} \PYG{l+m+mf}{1.2e\PYGZhy{}3}\PYG{p}{,} \PYG{l+s+s1}{\PYGZsq{}}\PYG{l+s+s1}{unit}\PYG{l+s+s1}{\PYGZsq{}}\PYG{p}{:} \PYG{l+s+s1}{\PYGZsq{}}\PYG{l+s+s1}{A}\PYG{l+s+s1}{\PYGZsq{}}\PYG{p}{\PYGZcb{}}\PYG{p}{,}
    \PYG{l+s+s1}{\PYGZsq{}}\PYG{l+s+s1}{Spectrum}\PYG{l+s+s1}{\PYGZsq{}}\PYG{p}{:} \PYG{p}{\PYGZob{}}\PYG{l+s+s1}{\PYGZsq{}}\PYG{l+s+s1}{value}\PYG{l+s+s1}{\PYGZsq{}}\PYG{p}{:} \PYG{n}{numpy}\PYG{o}{.}\PYG{n}{array}\PYG{p}{(}\PYG{p}{[}\PYG{o}{.}\PYG{o}{.}\PYG{o}{.}\PYG{p}{]}\PYG{p}{)}\PYG{p}{,} \PYG{l+s+s1}{\PYGZsq{}}\PYG{l+s+s1}{unit}\PYG{l+s+s1}{\PYGZsq{}}\PYG{p}{:} \PYG{l+s+s1}{\PYGZsq{}}\PYG{l+s+s1}{cnt}\PYG{l+s+s1}{\PYGZsq{}}\PYG{p}{\PYGZcb{}}
\PYG{p}{\PYGZcb{}}
\end{sphinxVerbatim}

\sphinxlineitem{export\_started (str):}
\sphinxAtStartPar
Wird ausgelöst, wenn eine neue Datei erstellt wurde.
Args: (str: Voller Pfad zur Datei).

\sphinxlineitem{export\_finished (str):}
\sphinxAtStartPar
Wird ausgelöst, wenn der Export beendet wurde.
Args: (str: Dateiname).

\sphinxlineitem{export\_error (str):}
\sphinxAtStartPar
Wird bei Schreib\sphinxhyphen{}/IO\sphinxhyphen{}Fehlern ausgelöst.
Args: (str: Fehlermeldung).

\end{description}

\end{description}
\index{\_\_init\_\_() (modules.export.ExportManager.ExportManager method)@\spxentry{\_\_init\_\_()}\spxextra{modules.export.ExportManager.ExportManager method}}

\begin{fulllineitems}
\phantomsection\label{\detokenize{modulab/export_manager:modules.export.ExportManager.ExportManager.__init__}}
\pysigstartsignatures
\pysiglinewithargsret
{\sphinxbfcode{\sphinxupquote{\_\_init\_\_}}}
{\sphinxparam{\DUrole{n}{log\_manager}}\sphinxparamcomma \sphinxparam{\DUrole{n}{profile\_manager}}}
{}
\pysigstopsignatures
\sphinxAtStartPar
Initialisiert den ExportManager.

\end{fulllineitems}

\index{select\_directory\_dialog() (modules.export.ExportManager.ExportManager method)@\spxentry{select\_directory\_dialog()}\spxextra{modules.export.ExportManager.ExportManager method}}

\begin{fulllineitems}
\phantomsection\label{\detokenize{modulab/export_manager:modules.export.ExportManager.ExportManager.select_directory_dialog}}
\pysigstartsignatures
\pysiglinewithargsret
{\sphinxbfcode{\sphinxupquote{select\_directory\_dialog}}}
{}
{}
\pysigstopsignatures
\sphinxAtStartPar
Öffnet einen System\sphinxhyphen{}Dialog zur Auswahl des Speicherordners.

\sphinxAtStartPar
\sphinxstylestrong{Thread\sphinxhyphen{}Safety:}
Diese Funktion erkennt automatisch, ob sie aus einem Worker\sphinxhyphen{}Thread (Experiment)
oder dem Haupt\sphinxhyphen{}Thread aufgerufen wird. Wenn sie aus einem Worker\sphinxhyphen{}Thread kommt,
pausiert sie diesen und führt den GUI\sphinxhyphen{}Dialog sicher im Haupt\sphinxhyphen{}Thread aus.

\sphinxAtStartPar
Der gewählte Pfad wird automatisch im Profil gespeichert (\sphinxtitleref{Export\_LastDir}).
\begin{quote}\begin{description}
\sphinxlineitem{Returns}
\sphinxAtStartPar
Der ausgewählte (oder vorherige) Pfad.

\sphinxlineitem{Return type}
\sphinxAtStartPar
str

\end{description}\end{quote}

\begin{sphinxadmonition}{note}{Examples}

\sphinxAtStartPar
Button\sphinxhyphen{}Click Handler in der GUI:

\begin{sphinxVerbatim}[commandchars=\\\{\}]
\PYG{k}{def}\PYG{+w}{ }\PYG{n+nf}{on\PYGZus{}btn\PYGZus{}browse\PYGZus{}clicked}\PYG{p}{(}\PYG{p}{)}\PYG{p}{:}
    \PYG{n}{new\PYGZus{}path} \PYG{o}{=} \PYG{n}{export\PYGZus{}mgr}\PYG{o}{.}\PYG{n}{select\PYGZus{}directory\PYGZus{}dialog}\PYG{p}{(}\PYG{p}{)}
    \PYG{n}{ui}\PYG{o}{.}\PYG{n}{line\PYGZus{}edit\PYGZus{}path}\PYG{o}{.}\PYG{n}{setText}\PYG{p}{(}\PYG{n}{new\PYGZus{}path}\PYG{p}{)}
\end{sphinxVerbatim}
\end{sphinxadmonition}

\end{fulllineitems}

\index{set\_export\_directory() (modules.export.ExportManager.ExportManager method)@\spxentry{set\_export\_directory()}\spxextra{modules.export.ExportManager.ExportManager method}}

\begin{fulllineitems}
\phantomsection\label{\detokenize{modulab/export_manager:modules.export.ExportManager.ExportManager.set_export_directory}}
\pysigstartsignatures
\pysiglinewithargsret
{\sphinxbfcode{\sphinxupquote{set\_export\_directory}}}
{\sphinxparam{\DUrole{n}{path}}}
{}
\pysigstopsignatures
\sphinxAtStartPar
Setzt den Export\sphinxhyphen{}Pfad manuell und speichert ihn im Profil.
\begin{quote}\begin{description}
\sphinxlineitem{Parameters}
\sphinxAtStartPar
\sphinxstyleliteralstrong{\sphinxupquote{path}} (\sphinxcode{\sphinxupquote{str}}) \textendash{} Der absolute Pfad zum Zielordner.

\end{description}\end{quote}

\end{fulllineitems}

\index{get\_export\_directory() (modules.export.ExportManager.ExportManager method)@\spxentry{get\_export\_directory()}\spxextra{modules.export.ExportManager.ExportManager method}}

\begin{fulllineitems}
\phantomsection\label{\detokenize{modulab/export_manager:modules.export.ExportManager.ExportManager.get_export_directory}}
\pysigstartsignatures
\pysiglinewithargsret
{\sphinxbfcode{\sphinxupquote{get\_export\_directory}}}
{}
{}
\pysigstopsignatures
\sphinxAtStartPar
Liest den aktuell konfigurierten Export\sphinxhyphen{}Pfad aus dem Profil.
\begin{quote}\begin{description}
\sphinxlineitem{Returns}
\sphinxAtStartPar
Der Pfad oder das Benutzer\sphinxhyphen{}Home\sphinxhyphen{}Verzeichnis als Fallback.

\sphinxlineitem{Return type}
\sphinxAtStartPar
str

\end{description}\end{quote}

\end{fulllineitems}

\index{new() (modules.export.ExportManager.ExportManager method)@\spxentry{new()}\spxextra{modules.export.ExportManager.ExportManager method}}

\begin{fulllineitems}
\phantomsection\label{\detokenize{modulab/export_manager:modules.export.ExportManager.ExportManager.new}}
\pysigstartsignatures
\pysiglinewithargsret
{\sphinxbfcode{\sphinxupquote{new}}}
{\sphinxparam{\DUrole{n}{filename\_base}}\sphinxparamcomma \sphinxparam{\DUrole{n}{dataset\_name}\DUrole{o}{=}\DUrole{default_value}{\textquotesingle{}Measurement\textquotesingle{}}}}
{}
\pysigstopsignatures
\sphinxAtStartPar
Erstellt eine neue HDF5\sphinxhyphen{}Datei und bereitet die Messung vor.

\sphinxAtStartPar
Der Dateiname wird automatisch mit einem Zeitstempel versehen
(\sphinxtitleref{Name\_YYYYMMDD\_HHMMSS.h5}). Schließt eine evtl. offene Datei zuvor.
\begin{quote}\begin{description}
\sphinxlineitem{Parameters}\begin{itemize}
\item {} 
\sphinxAtStartPar
\sphinxstyleliteralstrong{\sphinxupquote{filename\_base}} (\sphinxcode{\sphinxupquote{str}}) \textendash{} Der Basisname der Datei (z.B. “Experiment\_A”).

\item {} 
\sphinxAtStartPar
\sphinxstyleliteralstrong{\sphinxupquote{dataset\_name}} (\sphinxcode{\sphinxupquote{str}}) \textendash{} Der Name der HDF5\sphinxhyphen{}Gruppe für die Daten
(Standard: “Measurement”).

\end{itemize}

\sphinxlineitem{Returns}
\sphinxAtStartPar
True bei Erfolg, False bei IO\sphinxhyphen{}Fehlern.

\sphinxlineitem{Return type}
\sphinxAtStartPar
bool

\end{description}\end{quote}

\begin{sphinxadmonition}{note}{Examples}

\sphinxAtStartPar
Eine neue Messdatei starten:

\begin{sphinxVerbatim}[commandchars=\\\{\}]
\PYG{k}{if} \PYG{n}{export\PYGZus{}mgr}\PYG{o}{.}\PYG{n}{new}\PYG{p}{(}\PYG{l+s+s2}{\PYGZdq{}}\PYG{l+s+s2}{OLED\PYGZus{}IV\PYGZus{}Curve}\PYG{l+s+s2}{\PYGZdq{}}\PYG{p}{)}\PYG{p}{:}
    \PYG{n+nb}{print}\PYG{p}{(}\PYG{l+s+s2}{\PYGZdq{}}\PYG{l+s+s2}{Datei erstellt, bereit für Daten.}\PYG{l+s+s2}{\PYGZdq{}}\PYG{p}{)}
\end{sphinxVerbatim}
\end{sphinxadmonition}

\end{fulllineitems}

\index{add() (modules.export.ExportManager.ExportManager method)@\spxentry{add()}\spxextra{modules.export.ExportManager.ExportManager method}}

\begin{fulllineitems}
\phantomsection\label{\detokenize{modulab/export_manager:modules.export.ExportManager.ExportManager.add}}
\pysigstartsignatures
\pysiglinewithargsret
{\sphinxbfcode{\sphinxupquote{add}}}
{\sphinxparam{\DUrole{n}{name}}\sphinxparamcomma \sphinxparam{\DUrole{n}{data}}\sphinxparamcomma \sphinxparam{\DUrole{n}{unit}\DUrole{o}{=}\DUrole{default_value}{\textquotesingle{}\textquotesingle{}}}}
{}
\pysigstopsignatures
\sphinxAtStartPar
Fügt dynamische Messdaten zum internen Puffer hinzu.

\sphinxAtStartPar
Diese Daten werden \sphinxstylestrong{noch nicht} auf die Festplatte geschrieben.
Erst der Aufruf von \sphinxtitleref{commit()} schreibt alle mit \sphinxtitleref{add()} gesammelten
Werte als eine Zeile in die HDF5\sphinxhyphen{}Datasets.
\begin{quote}\begin{description}
\sphinxlineitem{Parameters}\begin{itemize}
\item {} 
\sphinxAtStartPar
\sphinxstyleliteralstrong{\sphinxupquote{name}} (\sphinxcode{\sphinxupquote{str}}) \textendash{} Der Name des Datasets (Spaltenname), z.B. “Voltage”.

\item {} 
\sphinxAtStartPar
\sphinxstyleliteralstrong{\sphinxupquote{data}} (\sphinxcode{\sphinxupquote{float | numpy.ndarray}}) \textendash{} Der Messwert (Skalar) oder ein Array
(z.B. ganzes Spektrum).

\item {} 
\sphinxAtStartPar
\sphinxstyleliteralstrong{\sphinxupquote{unit}} (\sphinxcode{\sphinxupquote{str}}, \sphinxstyleemphasis{optional}) \textendash{} Die physikalische Einheit (z.B. “V”, “nm”),
wird als HDF5\sphinxhyphen{}Attribut gespeichert.

\end{itemize}

\end{description}\end{quote}

\begin{sphinxadmonition}{note}{Examples}

\sphinxAtStartPar
Werte für den nächsten Zeitschritt sammeln:

\begin{sphinxVerbatim}[commandchars=\\\{\}]
\PYG{c+c1}{\PYGZsh{} Skalar hinzufügen}
\PYG{n}{export\PYGZus{}mgr}\PYG{o}{.}\PYG{n}{add}\PYG{p}{(}\PYG{l+s+s2}{\PYGZdq{}}\PYG{l+s+s2}{Time}\PYG{l+s+s2}{\PYGZdq{}}\PYG{p}{,} \PYG{l+m+mf}{1.5}\PYG{p}{,} \PYG{l+s+s2}{\PYGZdq{}}\PYG{l+s+s2}{s}\PYG{l+s+s2}{\PYGZdq{}}\PYG{p}{)}
\PYG{n}{export\PYGZus{}mgr}\PYG{o}{.}\PYG{n}{add}\PYG{p}{(}\PYG{l+s+s2}{\PYGZdq{}}\PYG{l+s+s2}{Current}\PYG{l+s+s2}{\PYGZdq{}}\PYG{p}{,} \PYG{l+m+mf}{1e\PYGZhy{}6}\PYG{p}{,} \PYG{l+s+s2}{\PYGZdq{}}\PYG{l+s+s2}{A}\PYG{l+s+s2}{\PYGZdq{}}\PYG{p}{)}

\PYG{c+c1}{\PYGZsh{} Array hinzufügen (z.B. Spektrum)}
\PYG{n}{spectrum\PYGZus{}data} \PYG{o}{=} \PYG{n}{np}\PYG{o}{.}\PYG{n}{array}\PYG{p}{(}\PYG{p}{[}\PYG{l+m+mi}{1}\PYG{p}{,} \PYG{l+m+mi}{2}\PYG{p}{,} \PYG{l+m+mi}{3}\PYG{p}{,} \PYG{o}{.}\PYG{o}{.}\PYG{o}{.}\PYG{p}{]}\PYG{p}{)}
\PYG{n}{export\PYGZus{}mgr}\PYG{o}{.}\PYG{n}{add}\PYG{p}{(}\PYG{l+s+s2}{\PYGZdq{}}\PYG{l+s+s2}{Spectrum}\PYG{l+s+s2}{\PYGZdq{}}\PYG{p}{,} \PYG{n}{spectrum\PYGZus{}data}\PYG{p}{,} \PYG{l+s+s2}{\PYGZdq{}}\PYG{l+s+s2}{counts}\PYG{l+s+s2}{\PYGZdq{}}\PYG{p}{)}

\PYG{c+c1}{\PYGZsh{} WICHTIG: Jetzt commit aufrufen!}
\PYG{n}{export\PYGZus{}mgr}\PYG{o}{.}\PYG{n}{commit}\PYG{p}{(}\PYG{p}{)}
\end{sphinxVerbatim}
\end{sphinxadmonition}

\end{fulllineitems}

\index{add\_static() (modules.export.ExportManager.ExportManager method)@\spxentry{add\_static()}\spxextra{modules.export.ExportManager.ExportManager method}}

\begin{fulllineitems}
\phantomsection\label{\detokenize{modulab/export_manager:modules.export.ExportManager.ExportManager.add_static}}
\pysigstartsignatures
\pysiglinewithargsret
{\sphinxbfcode{\sphinxupquote{add\_static}}}
{\sphinxparam{\DUrole{n}{name}}\sphinxparamcomma \sphinxparam{\DUrole{n}{data}}\sphinxparamcomma \sphinxparam{\DUrole{n}{unit}\DUrole{o}{=}\DUrole{default_value}{\textquotesingle{}\textquotesingle{}}}}
{}
\pysigstopsignatures
\sphinxAtStartPar
Speichert einmalige, statische Daten (Metadaten/Konstanten).

\sphinxAtStartPar
Im Gegensatz zu \sphinxtitleref{add()} wird hier \sphinxstylestrong{sofort} geschrieben.
Das Dataset hat die Länge 1 und wird nicht erweitert.
\begin{quote}\begin{description}
\sphinxlineitem{Parameters}\begin{itemize}
\item {} 
\sphinxAtStartPar
\sphinxstyleliteralstrong{\sphinxupquote{name}} (\sphinxcode{\sphinxupquote{str}}) \textendash{} Name des Datasets.

\item {} 
\sphinxAtStartPar
\sphinxstyleliteralstrong{\sphinxupquote{data}} \textendash{} Der Wert (String, Zahl, Array).

\item {} 
\sphinxAtStartPar
\sphinxstyleliteralstrong{\sphinxupquote{unit}} (\sphinxcode{\sphinxupquote{str}}, \sphinxstyleemphasis{optional}) \textendash{} Einheit.

\end{itemize}

\end{description}\end{quote}

\begin{sphinxadmonition}{note}{Examples}

\sphinxAtStartPar
Geräte\sphinxhyphen{}Infos speichern:

\begin{sphinxVerbatim}[commandchars=\\\{\}]
\PYG{n}{export\PYGZus{}mgr}\PYG{o}{.}\PYG{n}{add\PYGZus{}static}\PYG{p}{(}\PYG{l+s+s2}{\PYGZdq{}}\PYG{l+s+s2}{User}\PYG{l+s+s2}{\PYGZdq{}}\PYG{p}{,} \PYG{l+s+s2}{\PYGZdq{}}\PYG{l+s+s2}{Max Mustermann}\PYG{l+s+s2}{\PYGZdq{}}\PYG{p}{)}
\PYG{n}{export\PYGZus{}mgr}\PYG{o}{.}\PYG{n}{add\PYGZus{}static}\PYG{p}{(}\PYG{l+s+s2}{\PYGZdq{}}\PYG{l+s+s2}{IntegrationTime}\PYG{l+s+s2}{\PYGZdq{}}\PYG{p}{,} \PYG{l+m+mi}{100}\PYG{p}{,} \PYG{l+s+s2}{\PYGZdq{}}\PYG{l+s+s2}{ms}\PYG{l+s+s2}{\PYGZdq{}}\PYG{p}{)}
\end{sphinxVerbatim}
\end{sphinxadmonition}

\end{fulllineitems}

\index{add\_group\_attribute() (modules.export.ExportManager.ExportManager method)@\spxentry{add\_group\_attribute()}\spxextra{modules.export.ExportManager.ExportManager method}}

\begin{fulllineitems}
\phantomsection\label{\detokenize{modulab/export_manager:modules.export.ExportManager.ExportManager.add_group_attribute}}
\pysigstartsignatures
\pysiglinewithargsret
{\sphinxbfcode{\sphinxupquote{add\_group\_attribute}}}
{\sphinxparam{\DUrole{n}{key}}\sphinxparamcomma \sphinxparam{\DUrole{n}{value}}}
{}
\pysigstopsignatures
\sphinxAtStartPar
Fügt Metadaten als Attribute zur HDF5\sphinxhyphen{}Hauptgruppe hinzu.
\begin{quote}\begin{description}
\sphinxlineitem{Parameters}\begin{itemize}
\item {} 
\sphinxAtStartPar
\sphinxstyleliteralstrong{\sphinxupquote{key}} (\sphinxcode{\sphinxupquote{str}}) \textendash{} Attribut\sphinxhyphen{}Name.

\item {} 
\sphinxAtStartPar
\sphinxstyleliteralstrong{\sphinxupquote{value}} \textendash{} Attribut\sphinxhyphen{}Wert.

\end{itemize}

\end{description}\end{quote}

\end{fulllineitems}

\index{commit() (modules.export.ExportManager.ExportManager method)@\spxentry{commit()}\spxextra{modules.export.ExportManager.ExportManager method}}

\begin{fulllineitems}
\phantomsection\label{\detokenize{modulab/export_manager:modules.export.ExportManager.ExportManager.commit}}
\pysigstartsignatures
\pysiglinewithargsret
{\sphinxbfcode{\sphinxupquote{commit}}}
{}
{}
\pysigstopsignatures
\sphinxAtStartPar
Schließt den aktuellen Datenpunkt ab (Synchronisation).

\sphinxAtStartPar
Führt folgende Schritte aus:
1. Erstellt HDF5\sphinxhyphen{}Datasets für neue Spalten (falls nötig).
2. Erweitert alle Datasets um eine Zeile.
3. Schreibt die gepufferten Werte aus \sphinxtitleref{add()} in die neue Zeile.
4. Füllt fehlende Werte (falls \sphinxtitleref{add} für eine Spalte vergessen wurde) mit NaN.
5. Sendet das \sphinxtitleref{data\_committed}\sphinxhyphen{}Signal für Live\sphinxhyphen{}Plots.
6. Leert den Puffer für den nächsten Punkt.

\begin{sphinxadmonition}{note}{Examples}

\sphinxAtStartPar
Am Ende einer Messschleife aufrufen:

\begin{sphinxVerbatim}[commandchars=\\\{\}]
\PYG{k}{while} \PYG{n}{measuring}\PYG{p}{:}
    \PYG{n}{val} \PYG{o}{=} \PYG{n}{instrument}\PYG{o}{.}\PYG{n}{read}\PYG{p}{(}\PYG{p}{)}
    \PYG{n}{export\PYGZus{}mgr}\PYG{o}{.}\PYG{n}{add}\PYG{p}{(}\PYG{l+s+s2}{\PYGZdq{}}\PYG{l+s+s2}{Reading}\PYG{l+s+s2}{\PYGZdq{}}\PYG{p}{,} \PYG{n}{val}\PYG{p}{)}
    \PYG{n}{export\PYGZus{}mgr}\PYG{o}{.}\PYG{n}{commit}\PYG{p}{(}\PYG{p}{)} \PYG{c+c1}{\PYGZsh{} Schreibt auf Disk \PYGZam{} updated Plot}
\end{sphinxVerbatim}
\end{sphinxadmonition}

\end{fulllineitems}

\index{stop() (modules.export.ExportManager.ExportManager method)@\spxentry{stop()}\spxextra{modules.export.ExportManager.ExportManager method}}

\begin{fulllineitems}
\phantomsection\label{\detokenize{modulab/export_manager:modules.export.ExportManager.ExportManager.stop}}
\pysigstartsignatures
\pysiglinewithargsret
{\sphinxbfcode{\sphinxupquote{stop}}}
{}
{}
\pysigstopsignatures
\sphinxAtStartPar
Beendet den Export und schließt die HDF5\sphinxhyphen{}Datei sauber.

\sphinxAtStartPar
Sendet das \sphinxtitleref{export\_finished}\sphinxhyphen{}Signal.

\end{fulllineitems}


\end{fulllineitems}



\renewcommand{\indexname}{Python Module Index}
\begin{sphinxtheindex}
\let\bigletter\sphinxstyleindexlettergroup
\bigletter{m}
\item\relax\sphinxstyleindexentry{modules.device.DeviceManager}\sphinxstyleindexpageref{modulab/device_manager:\detokenize{module-modules.device.DeviceManager}}
\item\relax\sphinxstyleindexentry{modules.export.ExportManager}\sphinxstyleindexpageref{modulab/export_manager:\detokenize{module-modules.export.ExportManager}}
\item\relax\sphinxstyleindexentry{modules.log.LogManager}\sphinxstyleindexpageref{modulab/log_manager:\detokenize{module-modules.log.LogManager}}
\item\relax\sphinxstyleindexentry{modules.profile.ProfileManager}\sphinxstyleindexpageref{modulab/profile_manager:\detokenize{module-modules.profile.ProfileManager}}
\item\relax\sphinxstyleindexentry{modules.smu.SmuManager}\sphinxstyleindexpageref{modulab/smu_manager:\detokenize{module-modules.smu.SmuManager}}
\item\relax\sphinxstyleindexentry{modules.spectrometer.SpectrometerManager}\sphinxstyleindexpageref{modulab/spectrometer_manager:\detokenize{module-modules.spectrometer.SpectrometerManager}}
\end{sphinxtheindex}

\renewcommand{\indexname}{Index}
\printindex
\end{document}